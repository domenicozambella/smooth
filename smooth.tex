\documentclass[10pt,oneside, openany]{book}
% \usepackage[utf8]{inputenc}
\usepackage[T1]{fontenc} 
% \usepackage[dvips,paperheight=11cm, paperwidth=21cm, ,hmargin={9.5cm,0.5cm},vmargin={0.5cm,0.5cm}]{geometry}
\usepackage[a4paper, hmargin={4cm,4cm},vmargin={2.5cm,2.5cm}]{geometry}
\usepackage[unicode, colorlinks=true,bookmarksopen=false,linkcolor=blue,citecolor=red]{hyperref}
\usepackage{subfiles}
\usepackage{calc}
\usepackage{comment}
\usepackage{amssymb}
\usepackage{amsthm}
\usepackage{amsmath}
\usepackage{amsrefs}
\usepackage{titlesec}
\usepackage{titletoc}
\usepackage{dsfont}
\usepackage{euscript}
\usepackage{fourier-orns}
\usepackage{pxfonts}
\usepackage{newpxtext}
\usepackage{tgpagella}
\usepackage{palatino}
\usepackage[sc]{mathpazo} % add possibly `sc` and `osf` options
%\usepackage{eulervm}
%\usepackage[adobe-utopia]{mathdesign}
\usepackage{graphicx}
\usepackage{pgfplots}
\usepgfplotslibrary{polar}
\usetikzlibrary{shapes.geometric}
\usetikzlibrary{calc}
\pgfplotsset{width=10cm,compat=1.9}
\usepackage{tikz}
\usepackage{tikz-cd}
\tikzcdset{
arrow style=tikz,
diagrams={>=latex}
}


\linespread{1.2}
\setlength{\parindent}{0ex}
\setlength{\parskip}{.4\baselineskip}
\definecolor{blue}{rgb}{0, 0.1, 0.6}

\DeclareFontFamily{OT1}{pzc}{}
\DeclareFontShape{OT1}{pzc}{m}{it}{<-> s * [1.10] pzcmi7t}{}
\DeclareMathAlphabet{\mathpzc}{OT1}{pzc}{m}{it}


\newcommand{\mylabel}[1]{{\ssf{#1}}\hfill}
\renewenvironment{itemize}
  {\begin{list}{$\triangleright$}{%
   \setlength{\parskip}{0mm}
   \setlength{\topsep}{.4\baselineskip}
   \setlength{\rightmargin}{0mm}
   \setlength{\listparindent}{0mm}
   \setlength{\itemindent}{0mm}
   \setlength{\labelwidth}{2ex}
   \setlength{\itemsep}{.2\baselineskip}
   \setlength{\parsep}{1ex}
   \setlength{\partopsep}{0mm}
   \setlength{\labelsep}{1ex}
   \setlength{\leftmargin}{\labelwidth+\labelsep}
   \let\makelabel\mylabel}}{%
   \end{list}\vspace*{-1.3mm}}

\renewcommand*{\emph}[1]{%
   \smash{\tikz[baseline]\node[rectangle, fill=green!40, rounded corners, inner xsep=0.5ex, inner ysep=0.2ex, anchor=base, minimum height = 2.7ex]{#1};}}

\def\jj{\mbox{-}}
\def\E{\exists}
\def\A{\forall}
\def\mdot{\mathord\cdot}
\def\models{\vDash}
\def\pmodels{\mathrel{\models\kern-1.5ex\raisebox{.5ex}{*}}}
\def\notmodels{\nvDash}
\def\proves{\vdash}
\def\notproves{\nvdash}
\def\proves{\vdash}
\def\provesT{\mathrel{\mathord{\vdash}\hskip-1.13ex\raisebox{-.5ex}{\tiny$T$}}}
\def\ZZ{\mathds Z}
\def\NN{\mathds N}
\def\QQ{\mathds Q}
\def\RR{\mathds R}
\def\BB{\mathds B}
\def\CC{\mathds C}
\def\PP{\mathds P}
\def\Indicator{\mathds I}
\def\Ar{{\rm Ar}}
\def\dom{\mathop{\rm dom}}
\def\range{\mathop{\rm range}}
\def\supp{\mathop{\rm supp}}
\def\rank{\mathop{\rm rank}}
\def\dcl{\mathop{\rm dcl}}
\def\acl{\mathop{\rm acl}}
\def\rad{\mathop{\rm rad}}
\def\eq{{{\rm eq}}}
\def\ccl{{\rm ccl}}
\def\Th{\textrm{Th}}
\def\Diag{{\rm Diag}}
\def\atpmTh{\textrm{Th}_\atpm}
\def\Mod{\mathop{\rm Mod}}
\def\Indicator{{\mathds 1}}
\def\Rmod{{\mbox{\scriptsize $R$-mod}}}
\def\Aut{{\rm Aut\kern.0ex}}
\def\Def{{\rm Def\kern.0ex}}
\def\Brl{{\rm Brl\kern.0ex}}
\def\Autf{\mathord{\rm Aut\kern.15ex{f}\kern.15ex}}
\def\orbit{\O}
\def\oorbit{\mathpzc{o}}
\def\oorbitf{\mathpzc{of\!}}
\def\id{\textrm{id}}
\def\st{{\rm st}}
\def\tp{{\rm tp}}
\def\atpm{{\tiny\rm at^\pm}}
\def\qftp{\textrm{qf\mbox{-}tp}}
\def\attp{\textrm{at\mbox{-}tp}}
\def\atpmtp{\atpm\mbox{-}\textrm{tp}}
\def\Deltatp{\Delta\mbox{-}\textrm{tp}}
\def\pmDelta{\Delta\hskip-.3ex\raisebox{1ex}{\tiny$\pm$}}
\def\pmDeltatp{\noindent\pmDelta\hskip-.3ex\textrm{-tp}}
\def\EMtp{\textrm{{\small EM}\mbox{-}tp}}


\def\sm{\smallsetminus}
\def\atpmL{L_{\atpm}\hskip-.3ex}
\def\qfL{L_{\rm qf}}
\def\atL{L_{\tiny\rm at}\hskip-.3ex}
\def\simdiff{\mathop\vartriangle}
\def\IMP{\Rightarrow}
\def\PMI{\Leftarrow}
\def\IFF{\Leftrightarrow}
\def\NIFF{\nLeftrightarrow}
\def\imp{\rightarrow}
\def\pmi{\leftarrow}
\def\iff{\leftrightarrow}
\def\niff{\mathrel{{\leftrightarrow}\llap{\raisebox{-.1ex}{{\small$/$}}\hskip.5ex}}}
\def\nimp{\mathrel{{\rightarrow}\llap{\raisebox{-.1ex}{{\small$/$}}\hskip.5ex}}}
\def\nequiv{\mathrel{\mbox{$\equiv$\llap{{\small$/$}\hskip.3ex}}}}
\def\equivEM{\stackrel{\smash{\scalebox{.5}{\rm EM}}}{\equiv}}
\def\equivL{\stackrel{\smash{\scalebox{.5}{\rm L}}}{\equiv}}
\def\equivKP{\stackrel{\smash{\scalebox{.5}{\rm KP}}}{\equiv}}
\def\equivSh{\stackrel{\smash{\scalebox{.5}{\rm Sh}}}{\equiv}}
\def\dIFF{\IFF\hskip-2.2ex\smash{\raisebox{1.3ex}{\tiny def}}}
\def\deq{\mathrel{=\hskip-1.9ex\smash{\raisebox{1.2ex}{\tiny def}}}}
\def\swedge{\mathbin{\raisebox{.2ex}{\tiny$\mathbin\wedge$}}}
\def\svee{\mathbin{\raisebox{.2ex}{\tiny$\mathbin\vee$}}}

\def\bigsum{\mathop{\mbox{\large$\displaystyle\sum$}}}
\def\bigint{\mathop{\mbox{\large$\displaystyle\int$}}}


\def\hookdoubleheadrightarrow{\hookrightarrow\mathrel{\mspace{-15mu}}\rightarrow}

\def\P{\EuScript P}
\def\D{\EuScript D}
\def\Aa{\EuScript A}
\def\Ee{\EuScript E}
\def\X{\EuScript X}
\def\Y{\EuScript Y}
\def\Z{\EuScript Z}
\def\C{\EuScript C}
\def\U{\EuScript U}
%\def\H{\EuScript H}
\def\I{\EuScript I}
\def\V{\EuScript V}
\def\R{\EuScript R}
\def\F{\EuScript F}
\def\G{\EuScript G}
\def\B{\EuScript B}
\def\M{\EuScript M}
\def\Ll{\EuScript L}
\def\K{\EuScript K}
\def\O{\EuScript O}
\def\J{\EuScript J}
\def\S{\EuScript S}
\def\<{\langle}
\def\>{\rangle}
\def\0{\varnothing}
\def\theta{\vartheta}
\def\phi{\varphi}
\def\epsilon{\varepsilon}
\def\ssf#1{\textsf{\footnotesize #1}}

\titlecontents{section}
[3.8em] % ie, 1.5em (chapter) + 2.3em
{\vskip-1ex}
{\contentslabel{1.5em}}
{\hspace*{-2.3em}}
{\titlerule*[1pc]{}\contentspage}


\titleformat
{\chapter} % command
[display] % shape
{\bfseries\LARGE} % format
{Chapter \ \thechapter} % label
{0.5ex} % sep
{} % before-code
[] % after-code
\titleformat{\section}[block]{\Large\bfseries}{\makebox[5ex][r]{\textbf{\thesection}}}{1.5ex}{}
\titlespacing*{\chapter}{0em}{.5ex plus .2ex minus .2ex}{2.3ex plus .2ex}
\titlespacing*{\section}{-9.7ex}{3ex plus .5ex minus .5ex}{1ex plus .2ex minus .2ex}

\renewcommand*\thesection{\arabic{section}}

\newtheoremstyle{mio}% name
     {2\parskip}%      Space above
     {\parskip}%      Space below
     {\vl}%         Body font
     {}%         Indent amount (empty = no indent, \parindent = para indent)
     {\bfseries}% Thm head font
     {}%        Punctuation after thm head
     {1.5ex}%     Space after thm head: " " = normal interword space;
           %   \newline = linebreak
     {\llap{\thmnumber{\vl #2}\hskip2mm}\thmname{\vl #1}\thmnote{\kern1ex\bfseries\vl #3}}% Thm head spec (can be left empty, meaning `normal')

\newtheoremstyle{liscio}% name
     {2\parskip}%      Space above
     {0mm}%      Space below
     {}%         Body font
     {}%         Indent amount (empty = no indent, \parindent = para indent)
     {\bfseries}% Thm head font
     {}%        Punctuation after thm head
     {1.5ex}%     Space after thm head: " " = normal interword space;
           %   \newline = linebreak
     {\llap{\thmnumber{#2}\hskip2mm}\thmname{#1}\thmnote{\bfseries{} #3}}% Thm head spec (can be left empty, meaning `normal')



\newcounter{thm}[chapter]

% \renewcommand{\thethm}{\thechapter.\arabic{thm}}
\renewcommand{\thethm}{\arabic{thm}}
\theoremstyle{mio}
\newtheorem{theorem}[thm]{Theorem}
\newtheorem{corollary}[thm]{Corollary}
\newtheorem{proposition}[thm]{Proposition}
\newtheorem{lemma}[thm]{Lemma}
\newtheorem{fact}[thm]{Fact}
\newtheorem{assumption}[thm]{Assumption}
\newtheorem{void_thm}[thm]{\kern-1ex}
\theoremstyle{liscio}
\newtheorem{definition}[thm]{Definition}
\newtheorem{void_def}[thm]{\kern-1ex}
\newtheorem{remark}[thm]{Remark}
\newtheorem{notation}[thm]{Notation}
\newtheorem{conjecture}[thm]{Conjecture}
\newtheorem{note}[thm]{Note}
\newtheorem{exercise}[thm]{Exercise}
\newtheorem{example}[thm]{Example}
\newtheorem{question}[thm]{Question}
\setlength{\partopsep}{0mm}
\setlength{\topsep}{0mm}

\def\QED{\noindent\nolinebreak[4]\hfill\rlap{\ \ $\Box$}\medskip}
\renewenvironment{proof}[1][Proof]%
{\smallskip\begin{trivlist}\item[\hskip\labelsep {\bf #1}]}
{\QED\end{trivlist}}

\newenvironment{void}[1][]%
{\begin{trivlist}\item[\hskip\labelsep {\bf #1}]}
{\QED\end{trivlist}}



\pagestyle{plain}

\definecolor{violet}{RGB}{105, 5, 0}
\definecolor{brown}{RGB}{150, 50, 10}
\definecolor{green}{RGB}{5,100, 15}
\def\mr{\color{brown}}
\def\gr{\color{green}}
\def\vl{\color{violet}}


\def\mrA{{\mr\Aa}}
\def\mrB{{\mr\B}}
\def\mrC{{\mr\C}}
\def\mrD{{\mr\D}}
\def\grB{{\gr\B}}
\def\grC{{\gr\C}}
\def\grD{{\gr\D}}


% \thickmuskip=2mu plus 0.5mu minus 0.5mu
% \medmuskip=1mu plus 0.2mu minus 0.2mu
\def\vc{{\footnotesize VC}}
\def\nip{{\footnotesize NIP}}
\def\ip{{\footnotesize IP}}
\def\Fr{\mathop{\rm Fr}}
\def\Var{\mathop{\rm Var}}
\def\Ex{\mathord{\mathds E}}
\def\Pr{\mathord{\mathds P}}


\def\ns{{}^*\kern-.2ex}
\def\nso{{}^\circ\kern-.15ex}
\def\nsS{{}^*\kern-.3ex S}
\def\cnonfork{\mathbin{\raise1.8ex\rlap{\kern0.6ex\rule{0.6ex}{0.1ex}}\rlap{\kern1.1ex\rule{0.1ex}{1.9ex}}\raise-0.3ex\hbox{$\smile$} } }

\def\simfin{\mathrel{\raise-.8ex\rlap{\kern0.9ex$\scriptscriptstyle\NN$}\sim\kern.5ex }}

\def\lessfin{\mathrel{\raise-.8ex\rlap{\kern0.3ex$\scriptscriptstyle\NN$}\raise.2ex\hbox{$<$} }}

\def\grtfin{\mathrel{\raise-.8ex\rlap{\kern0.6ex$\scriptscriptstyle\NN$}\raise.2ex\hbox{$>$} }}


\def\simpoly{\mathrel{\raise-.1ex\rlap{\kern0.6ex\tiny p}\raise.2ex\hbox{$\sim$} }}

\def\lesspoly{\mathrel{\raise-.3ex\rlap{\kern0.4ex\tiny p}\raise.2ex\hbox{$<$} }}

\def\grtpoly{\mathrel{\raise-.3ex\rlap{\kern0.7ex\tiny p}\raise.2ex\hbox{$>$} }}


\def\simlin{\mathrel{\raise-.3ex\rlap{\kern0.6ex\tiny l}\raise.2ex\hbox{$\sim$} }}

\def\lesslin{\mathrel{\raise-.5ex\rlap{\kern0.5ex\tiny l}\raise.2ex\hbox{$<$} }}

\def\grtlin{\mathrel{\raise-.3ex\rlap{\kern0.7ex\tiny l}\raise.2ex\hbox{$>$} }}

\def\eqPr{\mathrel{\raise-.6ex\rlap{\kern0.3ex\tiny Pr}\raise.1ex\hbox{$=$\,} }}


\begin{document}
\raggedbottom
% \renewcommand{\contentsname}{Pseudofinite Yoga}
% \tableofcontents

\begin{comment}

  \chapter{Pseudofinite samples}
  
  
  \def\medrel#1{\parbox[t]{6ex}{\hfil$\displaystyle #1$}}
  \def\ceq#1#2#3{\parbox[t]{25ex}{$\displaystyle #1$}\medrel{#2}$\displaystyle  #3$}
  
  
  % Let $\U$, $\V$ and $\phi(x\,;z)$ be as in the notes. Question: is the following lemma true?
  
  % \begin{lemma}
  %   If $\phi(x\,;z)$ has \vc-dimension $n$ then there are $b\in\V$ and $A_0\subseteq\U$ such that for every $c\in\V$ 
    
  %   \ceq{\hfill\phi(A_0,b)=\phi(A_0;c)}{\IMP}{\phi(\U,b)=\phi(\U;c)}
  
  % moreover, $A_0$ is finite and its cardinality depends only on $n$.
  % \end{lemma}
  
  % Does Lemma~1 follows from Lemma~2 below? Probably not, but why not?
  
  % \begin{lemma}
  %   If $\phi(x\,;z)$ has \vc-dimension $n$ then for every finite $A\subseteq\U$ there are $b\in\V$ and $A_0\subseteq A$ such that for every $c\in\V$ 
    
  %   \ceq{\hfill\phi(A_0,b)=\phi(A_0;c)}{\IMP}{\phi(A,b)=\phi(A;c)}
  
  % moreover, the cardinality of $A_0$ depends only on $n$.
  % \end{lemma}
  
  
  \def\medrel#1{\parbox[t]{5ex}{\hfil$\displaystyle #1$}}
  \def\ceq#1#2#3{\parbox[t]{15ex}{$\displaystyle #1$}\medrel{#2}$\displaystyle  #3$}
  
  Fix a complete theory $T$ in the language $L$.
  %We write $\kappa$ for the cardinality of $\U$ and assume $\kappa$ to be an inaccessible cardinal larger than $|L|$.
  Let $M\models T$.  
  For every positive integer $n$ define 
  
  \ceq{\hfill F_n}{=}{\Big\{f:M^n\to\RR\ \ :\ \ fa=0\ \textrm{for all but finitely many }a\Big\}}
  
  We write $\bar M$ for the multi-sorted structure $\<M,\RR,F_1, F_2,\dots\>$.
  We need not specify right away the language of $\bar M$, for the moment we only need that the elementary map $\gamma:M\to N$ has an unique extension to an elementary map $\bar\gamma:\bar M\to\bar N$ such that
  \begin{itemize}
    \item[1.] $\bar\gamma\restriction M=\gamma$ 
    \item[2.] $\bar\gamma\restriction\RR\,=\id_\RR$
    \item[3.] $\bar\gamma(f) =f\circ\gamma$, for every $f\in F_n$.
  \end{itemize}

  The class of models $\bar M$ as above has the amalgamation property if as morphism we take the elementary maps.
  Let $\ns\bar\U$ be a rich (i.e.\@ universal-homogeneous) mo






  defined as follows that is and fix an elementary extension of it that is saturated and of cardinality $\kappa$.
  This will be be denoted by $\bar\U=\<\U,\ns\RR,\F_1, \F_2,\dots\>$. 
  Note that there is no loss of generality in assuming that $\U$ is the domain of the homesort of $\bar\U$.
  
  We will not be too precise about the language $\bar L$ of this expansion. 
  We assume that $\bar L$ expands $L$ and that it is large enough to accomodate all symbols used below.
  The only constraint is that $\vert\bar L\vert<\kappa$. 
  This is necessary to guarantee the existence of the required saturated extension $\bar\U$.
  
  \def\medrel#1{\parbox[t]{5ex}{\hfil$\displaystyle #1$}}
  \def\ceq#1#2#3{\parbox[t]{25ex}{$\displaystyle #1$}\medrel{#2}
  $\displaystyle  #3$}
  
  In order to present the notation, we now introduce a few symbol contained in $\bar L$.
  For every formula in $\phi(x,z)\in\bar L$ there is a symbol in $\bar L$ for what is interpreted as a function
  
    \ceq{\hfill F_{|x|}\times\U^{|z|}}{\to}{\RR}
  
    \ceq{\hfill  (f,b)\ \ }{\mapsto}{\sum_{\phi(x,b)} fx.}
  
  We may write $\mu_f\phi(x,b)$ for $\sum_{\phi(x,b)} fx$, to stress that the value of this fuction is interpreted as a finitely additive measure (with values in $\ns\RR$, for the moment). 
  
  The \emph{norm\/} of $f\in\F_{|x|}$ is defined to be
  
  \ceq{\hfill \emph{$\Vert f\Vert$}}{=}{\sum_{x=x} |fx|}.
  
  The \emph{support} of $f\in\F_{|x|}$ is the set \emph{$\supp f$} = $\{a\in\U^{|x|}:fa\neq0\}$.
  The (internal) cardinality of the supprt is denoted by \emph{$\vert \supp f\vert$.}
  In general, $\vert \supp f\vert$ is a hyperfinite integer. 
  If $a,b\in\F_{|x|}$ are $\{0,1\}$-valued we confused them with their support.
  E.g., we write $a\subseteq b$ for $\supp a\subseteq\supp b$.
  
  \begin{notation}
    Throughout the following \emph{$\Delta$\/} is a collection of definable subsets of $\U^{|x|}$ or, depending on the context, the collection of formulas defining these sets.\QED
  \end{notation}
  
  \begin{lemma}
    Let $\mu$ be finitely additive signed measures on $\Delta$, a Boolean algebra of small cardinality.
    Then there is $f\in\F_{|x|}$ such that 
    
    \ceq{\hfill\mu_f\phi(x)}{=}{\mu\phi(x)}\hfill for every $\phi(x)\in\Delta$.
       
  \end{lemma}
  
  \begin{proof}
  \def\ceq#1#2#3{\parbox[t]{15ex}{$\displaystyle #1$}\medrel{#2}$\displaystyle  #3$}
    Let $u$ be a variables of sort $\F_{|x|}$.
    We claim that the type $p(u)$ defined below is finitely consistent 
  
    \ceq{\hfill p(u)}{=}{\Big\{\sum_{\phi(x)}ux=\mu\phi(x)\quad:\ \ \phi(x)\in\Delta\Big\}}
  
    Let $\{\phi_1(x),\dots,\phi_n(x)\}\subseteq\Delta$.
    It suffices to show that there is $f\in\F_{|x|}$ such that
    
    \def\ceq#1#2#3{\parbox[t]{33ex}{$\displaystyle #1$}\medrel{#2}$\displaystyle  #3$}
  
    \ceq{\ssf{1.}\hfill\sum_{\phi_i(x)}fx}{=}{\mu\phi_i(x)}\hfill for $i=1,\dots,n$.
  
    Without loss of generality we can assume that $\{\phi_1(x),\dots,\phi_n(x)\}$ is a Boolean algebra with atoms $\phi_1(x),\dots,\phi_k(x)$ for some $k\le n$.
    Pick some $a_1,\dots,a_k\in\U^{|x|}$ such that $a_i\models\phi_i(x)$.
    Pick $f\in\F_{|x|}$ with support $\{a_1,\dots,a_k\}$ and such that
    
    \ceq{\hfill f(a_i)}{=}{\mu\phi_i(x)}\hfill  for $i=1,\dots,k$.
  
    Clearly \ssf{1} above is satisfied by the finite additivity of the measure.
  \end{proof}
  
  
  \section{Smoothness}

  Let $\Sigma\subseteq\Delta$ be Boolean algebras of subsets of $\U^{|x|}$. We say that $f\in\F_{|x|}$ is \emph{smooth\/} between $\Delta$ and $\Sigma$ if for every $\epsilon\in\RR^+$ and $\phi(x)\in\Delta$ there are $\psi_1,(x),\psi_2(x)\in\Sigma$ such that $\psi_1(x)\imp\phi(x)\imp\psi_2(x)$ and $\mu_f\big(\psi_2(x)\sm\psi_1(x)\big)<\epsilon$.

  \begin{conjecture}
    Let $\Sigma=L_x(M)$ and $\Delta=L_x(\U)$. The following are equivalent
    \begin{itemize}
      \item[1.] $\mu_f$ is definable over $M$ (see Definition 4.10 in \href{https://arxiv.org/abs/1811.02139}{K.Gannon}\,);
      \item[2.] $f$ is smooth between $\Sigma$ and $\Delta$.
    \end{itemize}
  \end{conjecture}




  
\chapter{}
%%%%%%%%%%%%%%%%%%%%%%%%%%%%
%%%%%%%%%%%%%%%%%%%%%%%%%%%%
%%%%%%%%%%%%%%%%%%%%%%%%%%%%
%%%%%%%%%%%%%%%%%%%%%%%%%%%%
%%%%%%%%%%%%%%%%%%%%%%%%%%%%
%%%%%%%%%%%%%%%%%%%%%%%%%%%%
%%%%%%%%%%%%%%%%%%%%%%%%%%%%
\section{Pseudofinite yoga}
\def\medrel#1{\parbox[t]{5ex}{\hfil$\displaystyle #1$}}
\def\ceq#1#2#3{\parbox[t]{20ex}{$\displaystyle #1$}\medrel{#2}$\displaystyle  #3$}

We say that a sentence holds in \emph{almost all finite models\/} if it holds in every finite model of sufficienly large cardinality.

Let $T_0$ be a given theory with arbitrarily large finite models.
%
We define 

\ceq{\hfill\emph{$T_{\rm fin}$}}{=}{\{\phi: M\models\phi\textrm{ for almost all finite } M\models T_0\}.}

A structure $N$ is a \emph{pseudofinite\/} model of $T_0$ if $N\models T_{\rm fin}$.
%
This is equivalent to requiring that every sentence $\phi\in\Th(N)$ holds in arbitrarily large finite models of $T_0$.
%
Note that every model of $T_{\rm fin}$ is infinite.

We now discribe a canonical expansion of a finite model $M$ to a two sorted structure $\<M,\NN\>$.
%
The language of $\<M,\NN\>$ is denoted by $L'$.
\begin{itemize}
  \item[1.] $L'$ expands the language of $M$ and $\NN$, where $\NN$ is considered as a structure in the language with symbols for all real relations and functions of any finite arity.
  \item[2.] $L'$ has a function $f_\phi:M^{|z|}\to\NN$ for each formula $\phi(x\,;z)\in L'$ and every finite tuples of variables $x\,;z$ of the first sort. 
  %
  The interpretation of $f_\phi(b)$ is the (finite) cardinality of $\phi(M\,;b)$.
\end{itemize}

For some application a richer language will be necessary (cfr.\@ Section~\ref{EH} below).
%
Other times we need to expand $M$ with a few more sorts.

We define

\ceq{\hfill\emph{$T'_{\rm fin}$}}{=}{\{\phi\in L': \<M,\NN\>\models\phi\textrm{ for all finite } M\models T_0\}}.

% Note that, for any given $M$, the interpretation of $A_{\phi}$ is not unique.
% Therefore the expansion $\<M,\NN\>$ is not unique.
% %
% We read $\<M,\NN\>\models\phi$ as claiming that every expansion as above models $\phi$.

\bigskip
In the sequel \emph{$\<\U,\ns\NN\>$\/} denotes some arbitrary saturated model of $T'_{\rm fin}$.

\begin{fact}
  Assume that $\psi\in L'$ holds in every $\<\U,\ns\NN\>$ as above.
  %
  Then $\<M,\NN\>\models\psi$ for all sufficienly large finite $M\models T_0$.\QED 
\end{fact}

Below we write \emph{$\ns|\phi(x\,;b)|$\/} for $f_\phi(b)$.
%
We call this the \emph{pseudofinite cardinality\/} of $\phi(x\,;b)$.
%
By the definition of  $T'_{\rm fin}$ it is clear that the pseudofinite cardinality of a definable set $\D$ does not depend on the formula defining it, so we can unambiguously write $\ns|\D|$.

We will need two preorder relations (a.k.a.\@ quasiorders) on ${}^*\kern-.2ex\NN\sm\{0,1\}$. We say that the fist \emph{linear\/} the second \emph{polynomial.} Section~\ref{EH} only requires the second one. For $r,s\in {}^*\kern-.2ex\NN\sm\{0,1\}$
\begin{itemize}
  \item[1.]we write \emph{$r\leq_{\rm l }s$\/} if $r\le n\,s$ for some positive $n\in\NN$.
  \item[2.]we write \emph{$r\leq_{\rm p} s$\/} if $r\le s^n$ for some positive $n\in\NN$.
\end{itemize} 
The associated equivalence relation are denoted by\emph{$r\simlin s$\/} and \emph{$r\simpoly s$\/ }. 
%
Also, we write \emph{$r\lesslin s$\/} and \emph{$r\lesspoly s$\/} for the associated strict order, i.e.\@ if $n\,r<s$, respectively $r^n< s$ for all $n\in\NN$.

\begin{proposition}\label{prop_+=max}
  Let $\sim$ denote either $\simlin$ or $\simpoly$. 
  %
  For every non negative $r,s\in{}^*\kern-.2ex\NN$

  \ceq{\hfill r+s}
  {\sim}
  {\max\{r,\,s\}}

  In particular, if $\D, \C\subseteq\U^{z}$ are two definable sets, then

  \ceq{\hfill {}^*\kern-.2ex\big|\D\cup\C\big|}
  {\sim}
  {\max\big\{{}^*\kern-.2ex|\D|,\, {}^*\kern-.2ex|\C|\big\}.}
\end{proposition} 

\begin{proof}
  In fact 

\ceq{\hfill\max\{r,s\}}
{\le}
{r+s}

\ceq{}
{\le}
{2\max\{r,s\}}

\ceq{}
{\le}
{\big(\max\{r,s\}\big)^2}


This proves the first equivalence.
%
The second equivalence follows.
\end{proof}

Definable sets $\D\subseteq\U^x$ such that  $\ns|\D|\sim\ns|\U^x|$ are called \emph{large,} otherwise we say they are \emph{small.}
%
The context will clarify if we are referring to the linear or to the polynomial equivalence.


\begin{corollary}
  Let $\sim$ denote either $\simlin$ or $\simpoly$.
  %
  Let $p(x)\subseteq L(\U)$ be a type such that $\phi(\U)$ is large for every $\phi(x)$ that is conjunction of formulas in $p(x)$.
  %
  Then $p(x)$ has an extension to a complete type with the same property.\QED
\end{corollary}

The following fact is immediate. It is stated to illustrate the typical application of the notions introduced above.

\begin{fact}\label{fact_application}
  Let $\sim$ denote either $\simlin$ or $\simpoly$.
  %
  Let $\phi(x\,;z)\in L$, where $x$ has finite length.
  % 
  Assume that in every model $\<\U,\ns\NN\>$ there is a $b\in\U^{z}$ such that $\phi(\U^x\,;b)$ is large.
  %
  Then there is an $n\in\NN$ such that every sufficiently large finite $M\models T_0$ contains a tuple $c$ such that

  \ceq{\hfill\big|M^x\big|}{\le}{n\,\big|\phi(M^x\,;c)\big|}\hfill if $\sim$ is $\simlin$\phantom{.}\kern25ex
  
  \ceq{\hfill\big|M^x\big|}{\le}{\big|\psi(M^x\,;c)\big|^n}\hfill if $\sim$ is $\simpoly$.\kern25ex\rlap{$\square$}

\end{fact}

The following two proposition only hold for $\simpoly$.

\begin{proposition}\label{prop_+=x}
  For every non negative $r,s\in{}^*\kern-.2ex\NN$

  \ceq{\hfill r\cdot s}
  {\simpoly}
  {\max\{r,s\}}
\end{proposition} 

\begin{proof}
  In fact 

  \ceq{\hfill\max\{r,s\}}
  {\le}
  {r\cdot s}
  
  \ceq{}
  {\le}
  {\big(\max\{r,s\}\big)^2}
\end{proof}

Note that, in particular, $\ns|\U|\simpoly\ns|\U^x|$ for all tuples of variables of finite arity.


\def\ceq#1#2#3{\parbox[t]{25ex}{$\displaystyle #1$}\medrel{#2}$\displaystyle  #3$}


\begin{corollary}\label{coroll_union_small}
  Let $\psi(x,z)\in L'$, where $x,z$ are finite tuples of variables of the first sort.
  %
  Let $\Aa\subseteq\U$.
  %
  Assume that 
  
  \ceq{\hfill\ns|\Aa|}{\lesspoly}{\ns|\U|}
  
  \ceq{\hfill\ns|\psi(a,\U^z)|}{\lesspoly}{\ns|\U|}\quad for every $a\in\Aa$.
  
  Then 
  
  \ceq{\hfill\strut^*\!\Big|\bigcup_{a\in\Aa}\psi(a,\U^z)\Big|}{\lesspoly}{\ns|\U|}.
\end{corollary}

\begin{proof}
  Let $r=\ns|\Aa|$.
  %
  Let $s=\ns|\psi(a,z)|$, where $a\in\Aa$ is choosen such that $s$ is maximal.
  %
  Then 
  
  \ceq{\hfill\strut^*\!\Big|\bigcup_{a\in\Aa}\psi(a,\U^z)\Big|}{\le}{r\cdot s}


  \ceq{}{\lesspoly}{\ns|\U|\cdot\ns|\U|}

  \ceq{}{\simpoly}{\ns|\U|.}
\end{proof}


%%%%%%%%%%%%%%%%%%
%%%%%%%%%%%%%%%%%%
%%%%%%%%%%%%%%%%%%
%%%%%%%%%%%%%%%%%%
\section{Erd\H{o}s-Hajnal for stable graphs}\label{EH}
\def\ceq#1#2#3{\parbox[t]{15ex}{$\displaystyle #1$}\medrel{#2}$\displaystyle  #3$}
In this section we assume that the language $L'$ contains a few more symbols (below, for convenience, all finite models considered have as domain a subset of $\omega$).
%
\begin{itemize}
  \item[3.]  Let $X$ be an $n$-ary relation symbol of the first sort. 
  %
  Let $\phi\in L\cup\{X\}$ be a sentence.
  %
  Then we require that $L'$ has an $n$-ary predicate $A_\phi$ in the first sort.
  %
  The interpretation of $A_\phi$ is a relation of maximal cardinality that makes $\phi$ true in $M$.
  %
  If more then one relation satisfy the requirement above, we choose the minimal one in the lexicographic order (recall that the domain of $M$ is a subset of $\omega$).
\end{itemize}
Let $r(\mbox{-},\mbox{-})$ be a binary relation symbol.
%
The theory that $T_0$ says that $r(\mbox{-},\mbox{-})$ is an irreflexive and symmetric relation (i.e.\@ a graph) and that it is $m$-stable, where $m$ is fixed but arbitrary.
For the purpose of these notes $m$-stable means that $R_\Delta(\U)\le m$, where \emph{$R_\Delta$\/} is defined below.

Let \emph{$\Delta$} $=\{r(a,x) :\, a\in\U\}$.
%
Let $\S\subseteq\U$.
%
We denote by $R_\Delta\big(\S\big)$ the Shelah binary rank of $\S$.
%
That is, the maximal height $n$ of binary tree of $\pmDelta$-formulas $\<\phi_i(x)\,:\, s\in {}^{<n}2\>$ such that for every $s\in {}^{n}2$ the type 

\ceq{\hfill }
{}
{\big\{\neg^{s_i}\,\phi_{s\restriction i}(x)\, :\, i\le n\big\}}

has a solution in $\S$.

As $T_0$ is stable, $R_\Delta\big(\U\big)$ is finite.
%
Among the large $L(\U)$-definable subsets of $\U$ we fix one, say \emph{$\S=\sigma(\U)$}, such that $R_\Delta\big(\S\big)$ is minimal.

\begin{fact}
  For every $\pmDelta$-definable set $\D\subseteq\U$, exactly one between $\S\cap\D$ and $\S\cap\neg\D$ is large.
\end{fact}

\begin{proof}
  As $\S$ is large, at least one between $\S\cap\D$ and $\S\cap\neg\D$ is large by Proposition~\ref{prop_+=max}.
  Now, suppose for a contradiction that they both are large.
  Let $n=R_\Delta\big(\S\big)$.
  By the minimality of $n$, both $\S\cap\D$ and $\S\cap\neg\D$ have rank $n$.
  This contradicts the definition of rank.
\end{proof}

\begin{theorem}
  There is a large set $\Aa\subseteq\U$ that is either a clique or an anticlique.
\end{theorem}

\begin{proof}
  Let $\S=\sigma(\U)$ be as defined above.
  %
  Let $p(x)\subseteq\pmDelta$ be maximally consistent among the $\pmDelta$-types such that $\sigma(x)\wedge\phi(x)$ is large for every conjunction of formulas in the type.
  %
  By the fact above, $p(x)$ is $\pmDelta$-complete, i.e.\@ either $\phi(x)$ or $\neg\phi(x)$ is in $p(x)$ for every $\phi(x)\in\pmDelta$.

  % Moreover, by compactness, $R_\Delta\big(\S\big)=R_\Delta\big(p(\U)\cap\S\big)$.

  Let $\B=\{a\in\S\ :\ r(a,x)\in p\}$.
  %
  By stability, $\B$ is definable in $L(\U)$.
  %
  At least one between $\B$ and $\S\sm\B$ is large.
  %
  Assume the first, the argument when $\S\sm\B$ is large is similar.

  Let $\Aa$ be a definable subsets of $\B$ that is maximal among those that satisfy the formula $\big(\A x,y\in\Aa\big)\,r(x,y)$.
  %
  The definition of $L'$ guarantees that such a set exists.
  
  We claim that $\Aa$ is large, hence it is the clique required by the theorem.
  %
  So, assume not, and reason for a contradiction.

  By the maximality of $\Aa$, for every $b\in\B\sm\Aa$ there is an $a\in\Aa$ such that $\neg r(a,b)$.
  %
  We rephrase this as the inclusion

  \ceq{\hfill\B\sm\Aa}
  {\subseteq}
  {\bigcup_{a\in\Aa}\neg r(a,\S)}F

  By assumption $\Aa$ is not large, hence $\B\sm\Aa$ is large.
  %
  When $a\in\Aa$, in particular $r(a,x)\in p$.
  Therefore, by the fact above, $r(a,\S)$ is large and $\neg r(a,\S)$ is small.
  This is a contradiction by Corollary~\ref{coroll_union_small}.
\end{proof}

Finally the Erd\H{o}s-Hajnal property is obtained applying Fact~\ref{fact_application}.

\begin{corollary}
  There is an $n$ such that in every finite model $M\models T_0$ there is a set $A\subseteq M$ of cardinality at least $|M|^{1/n}$ that is either a clique or an anticlique.\QED
\end{corollary}


%%%%%%%%%%%%%%%%%%%%%%%%%%%%
%%%%%%%%%%%%%%%%%%%%%%%%%%%%
%%%%%%%%%%%%%%%%%%%%%%%%%%%%
%%%%%%%%%%%%%%%%%%%%%%%%%%%%
%%%%%%%%%%%%%%%%%%%%%%%%%%%%
\section{Pseudofinite samples}

\def\medrel#1{\parbox[t]{5ex}{\hfil$\displaystyle #1$}}
\def\ceq#1#2#3{\parbox[t]{25ex}{$\displaystyle #1$}\medrel{#2}$\displaystyle  #3$}
We now describe an expansion of a finite model $M$ to a $3$-sorted structure $\<M,\RR^M,\RR\>$, whose language is denoted by $L'$.
\begin{itemize}
  \item[1.] $L'$ expands the language of $M$ and $\RR$, where $\RR$ is considered as a structure in the language with symbols for all relations and functions of any finite arity.
  \item[2.] $L'$ expands the language of $\RR^M$, the set of function from $M$ to $\RR$, which we endow with the structure of $\RR$-algebra with the natural partial order.
  
  \item[3.] $L'$ contains a binary function $\RR^M\times M \to \RR$ for the evaluation of $f$, in $a$.
  
  \item[4.] $L'$ has a function $\Vert{\cdot}\Vert:\RR^M\to\RR$ which is interpreted as 
  
  \noindent\hskip-\labelwidth\hskip-\labelsep
  \ceq{\hfill\Vert f\Vert}{=}{\sum_{a\in M} |fa|.}
\end{itemize}

Subsets of $M$ are coded into this structure via their indicator function and their cardinality is definable via the function $\Vert{\cdot}\Vert$.

We will need the following easy lemma.

\begin{lemma}
  Fix $\epsilon\in\RR^+$ and $n\in\ZZ^+$.
  Then for every $g,h\in\RR^M$ such that $|h|<n{\cdot}g$ there is a function $f\in\RR^M$ that takes at most $n/\epsilon$ distinct values and such that $\Vert h-f{\cdot} g\Vert<\epsilon{\cdot}\Vert g\Vert$.
\end{lemma}

\begin{proof}
  The inequality $\Vert h-f{\cdot} g\Vert\le\epsilon{\cdot}\Vert g\Vert$ can be rewitten as $\Ex\big[h/g - f\big]\le\epsilon$, where $\Ex[\cdot]$ is the expectation w.r.t.\@ the probability mass function $g/\Vert g\Vert$.

  Let $m>0$ be an integer which will be specified below.
  For $i\in\ZZ$ we write $I_i$ for the real interval $[i{\cdot}n/m,\ (i+1){\cdot}n/m)$.
  Let $Y_i= (h/g)^{-1}[I_i]$.
  Finally define $f:M\to\RR$ by setting $fx=\Ex[h/g\, |\, Y_i]$, where $Y_i$ is is the unique set containing $x$.
  By assumption $-n<h/g<n$, hence $f$ takes at most $2m$ values.

  Note that for all random variables $X\in[a,b)$ we have  $\Ex\big[\big|X-\Ex[X]\big|\big]< (b-a)/2$.

  \ceq{\hfill\Ex\Big[\big|h/g-f\big|\ \Big|\ Y_i\Big]}
  {<}
  {n/2m}.

  As $Y_i$ is a partition of $M$ we obtain

  \ceq{\hfill\Ex\Big[\big|h/g-f\big|\Big]}
  {<}
  {n/2m}.

  Setting $m=\lceil n/2\epsilon\rceil$, we obtain the desired inequalities.  
\end{proof}


\begin{lemma}
  Fix $\epsilon\in\RR^+$ and $n\in\ZZ^+$.
  Then for every $g,h\in\RR^{M^2}$ such that $|h|<n{\cdot}g$ there is a function $f\in\RR^M$ that takes at most $n^2/\epsilon$ distinct values and such that $\Vert h-f{\cdot} g\Vert<\epsilon{\cdot}\Vert g\Vert$.
\end{lemma}

\begin{proof}
  Similar
\end{proof}

%%%%%%%%%%%%%%%%%%%%%%%%%%%%%%%%%%%%%%%%
%%%%%%%%%%%%%%%%%%%%%%%%%%%%%%%%%%%%%%%%
%%%%%%%%%%%%%%%%%%%%%%%%%%%%%%%%%%%%%%%%
\section{Szemer\'edi's regularity Lemma}
\def\ceq#1#2#3{\parbox[t]{40ex}{$\displaystyle #1$}\medrel{#2}$\displaystyle  #3$}

A pair of subsets $V',V''\subseteq M$ is \emph{$\epsilon$-regular\/} if for every $A\subseteq V'$ and $B\subseteq V''$

\ceq{\hfill\big|\Pr(r\,|\,A{\times} B) - \Pr(r\,|\,V'{\times} V'')\big|}
{<}
{\epsilon\, / \Pr(A{\times} B\,|\,V'{\times} V'')}

where $\Pr(\cdot)$ denotes the uniform probability measure on $M^2$.

\begin{void_thm}[Szemer\'edi's regularity Lemma]
  For every $\epsilon>0$ there is $k\in \NN$ such that the following obtains. For every graph $M$ of cardinality $>k$, there is a partition $\{V_i:i<k\}$ of $M$ such that

  \ceq{\hfill\sum_{\<i,j\>\in\Sigma}\Pr(V_i\times V_j)}
  {<}
  {\epsilon}

  where $\Sigma=\{\<i,j\> : V_i, V_j \textrm{ is not } \epsilon\textrm{-regular}\}$.
\end{void_thm}


\section{}





We write $\Delta$ for the set $\{r(x\,;b):b\in M\}$.
We write $\Delta_{\restriction A}$ if the parameters are restricted to range over $A\subseteq M$.

\begin{itemize}
  \item[5.] Contains a function for conditional expectation $\Ex:\RR^M\times\RR^M\to\RR^M$ that has the following interpretation
  
  \noindent\hskip-\labelwidth\hskip-\labelsep
  \ceq{\hfill\Ex[f\vert X]}
  {:}
  {M\ \ \to\ \ \RR}

  \noindent\hskip-\labelwidth\hskip-\labelsep
  \ceq{}{}
  {\ c\ \ \ \mapsto\ \frac{1}{\Vert Y\Vert}\sum_{a\in Y}fa} 
  
  \smallskip
  where $Y$ is the unique atom of $\Delta_{\restriction X}$ containing $c$.
\end{itemize}






Note that in $M$ a uniform comprehension principle holds.
I.e.\@, for every $\phi(x\,;z)\in L'$, where $x$ is a variable of the home sort, $Y$ a variable of sort $\RR^M$, and $z$ is a tuple of mixed sort,

\ceq{\hfill\A z\;\E Y\; \A x\ \ Yx}
{=}
{\left\{\begin{array}[]{llr}
  1&\textrm{if}&\phi(x,z)\\
  0&\textrm{if}&\neg\phi(x,z)\rlap{.}
\end{array}\right.}


Note that $T_0$ contains the following uniform comprehension principle.
I.e.\@, for every $\phi(x\,;z)\in L'$, where $x$ is a variable of the home sort, $f$ a variable of sort $\RR^M$, and $z$ is a tuple of mixed sort,

It follows that the caracteristic function of the set $\phi(M\,;b)$ is definable uniformly in $b$.
To streemline notation, when suggested by the context $\phi(x\,;z)$ denotes a family of caracteristic functions parametrized by $z$.

Fix a finite model $M$ and a formula $\phi(x\,;z)\in L'$, where $x$ and $z$ are both single variables.
We write $\Delta$ for the set $\{\phi(x\,;b):b\in M\}$.
We write $\Delta^\pm$ for the set $\Delta\cup\neg\Delta$, where $\neg\Delta$  is the set of negations of formulas in $\Delta$.

We write $\wedge\Delta^\pm$ for the set containing all conjunctions of $|M|$ formulas in $\Delta^\pm$.
This is the set containg the formulas

\ceq{\hfill\psi(x\,;f)}{=}{\A z\,\big[fz\neq0\ \iff\ \phi(x\,;z)\big]} 

as $f$ ranges in $\RR^M$.\le

%%%%%%%%%%%%%%%%%%%%%%%%%%%%
%%%%%%%%%%%%%%%%%%%%%%%%%%%%
%%%%%%%%%%%%%%%%%%%%%%%%%%%%
%%%%%%%%%%%%%%%%%%%%%%%%%%%%
%%%%%%%%%%%%%%%%%%%%%%%%%%%%
\section{Variance}



\begin{lemma}
  Let $B$ be a Bernoulli random variable.
  Let $F$ be a real valued random variable such that $\vert F\vert\le a$ for some $a$.
  Then

  \ceq{\hfill \Var(F)}{\le}{\Ex\big[\Var(F\,|\,B)\big]+a^2}
\end{lemma}

\begin{proof}
  In general, for any two random variables $F$ and $B$

  \ceq{\ssf1\hfill \Var(F)}{=}{\Ex\big[\Var(F\,|\,B)\big]+\Var\big(\Ex[F\,|\,B]\big)}

  Now, let $B$ be Bernoulli, and set $p=\Pr(B{=}1)$

  \ceq{\hfill\Var\big(\Ex[F\,|\,B]\big)}{=}{\Big(\Ex[F\,|\,B{=}0]-\Ex[F\,|\,B{=}1]\Big)^2 p\,(1-p).}

  Now, assuming $\vert F\vert\le a$ we obtain

  \ceq{}{\le}{(2a)^2  p\,(1-p).}

  Noting that $p\,(1-p)\le1/4$, we obtain

  \ceq{}{\le}{a^2}

  Hence the lemma follows from \ssf1.
\end{proof}

%%%%%%%%%%%%%%%%%%%%%%%%%%%%
%%%%%%%%%%%%%%%%%%%%%%%%%%%%
%%%%%%%%%%%%%%%%%%%%%%%%%%%%
%%%%%%%%%%%%%%%%%%%%%%%%%%%%
%%%%%%%%%%%%%%%%%%%%%%%%%%%%
\section{aborto}

Fix a finite model $M$.
Fix a formula $\phi(x\,;z)\in L'$.
In the lemma below, we write $\Delta$ for the set $\{\phi(x\,;b):b\in M\}$.
Then $\{\wedge\} \Delta^\pm$ denotes the set of conjunctions of formulas in $\Delta^\pm$. 
When we write $\psi(x\,;c) \in \{\wedge\} \Delta^\pm$ it is understood that the parameters in $\psi(x\,;c)$ are among $c=c_1,\dots,c_n$, where $c_i\in M^z$.
We say that $\psi(x\,;c)$ is complete if for all $i=1,\dots,n$ either $\phi(x\,;c_i)$ or its negation occurs in the conjunction.

\def\medrel#1{\parbox[t]{5ex}{\hfil$\displaystyle #1$}}
\def\ceq#1#2#3{\parbox[t]{35ex}{$\displaystyle #1$}\medrel{#2}$\displaystyle  #3$}
\begin{lemma}\label{lem_uno}
  For every $f \in \RR^M$ and every $\epsilon\in \RR^+$ there is a set $\Sigma\subseteq \{\wedge\} \Delta^\pm$ such that for every $b\in M^z$ exactly one of the following two inequalities holds
  
  \ceq{\hfill\big\Vert f \cdot \phi(x\,;b) \cdot \psi(x)\big)\Vert}{\le}{\epsilon}
  \\[-.5ex]
  \ssf1.
  \\[-.5ex]
  \ceq{\hfill\big\Vert  f \cdot \neg\phi(x\,;b) \cdot \psi(x)\big\Vert}{\le}{\epsilon.}
\end{lemma}



\begin{lemma}\label{lem_uno}
  For every $f \in \RR^M$ and every $\epsilon\in \RR^+$ there is an $n$ and a $c\in\big(M^{z}\big)^n$ such that for every complete formula $\psi(x\,;c) \in \{\wedge\} \Delta^\pm$ and every $b\in M^z$ one of the following two inequalities holds
  
  \ceq{\hfill\big\Vert f \cdot \phi(x\,;b) \cdot \psi(x\,;c\big)\Vert}{\le}{\epsilon}
  \\[-.5ex]
  \ssf1.
  \\[-.5ex]
  \ceq{\hfill\big\Vert  f \cdot \neg\phi(x\,;b) \cdot \psi(x\,;c)\big\Vert}{\le}{\epsilon.}
\end{lemma}

\begin{proof}
  Let $f$ and $\epsilon$ be given.
  Let $\gamma$ be the maximal integer such that, for some $c$, there is set $\Gamma$ of cardinality $\gamma$ containing pairwise inconsistent formulas $\psi(x\,;c)\in\{\wedge\}\Delta^\pm$ such that $\Vert f\cdot\psi(x\,;c)\big\Vert>\epsilon$.

  We claim that any $c$ as above proves the lemma.
  Let $\psi(x\,;c)\in\{\wedge\}\Delta^\pm$ be complete and suppose for a contradiction that for some  $b\in M^z$ neither one of the two inequalities in \ssf1 is satisfied. 
  That is

  \ceq{\hfill\big\Vert f\cdot\phi(x\,;b) \cdot \psi(x\,;c)\big]\big\Vert}
  {>}{\epsilon}
  \\[-.5ex]
  and
  \\[-.5ex]
  \ceq{\hfill\big\Vert f\cdot\neg\phi(x\,;b) \cdot \psi(x\,;c)\big]\big\Vert}
  {>}{\epsilon.}

  As in particular $\Vert f\cdot\psi(x\,;c)\Vert>\epsilon$, by the completeness of $\psi(x\,;c)$, the formulas in $
\Gamma$ are either inconsistent with or a consequence of  $\psi(x\,;c)$. Hence by the maximality of $\Gamma$, there is some formula $\psi'(x\,;c)\in\Gamma$ such that  $\psi(x\,;c)\imp\psi'(x\,;c)$. Then the set

  \hfil $\Big(\Gamma \sm \big\{\psi'(x\,;c)\big\}\Big)  \cup \Big\{ \phi(x\,;b)\wedge\psi(x\,;c),\ \  \neg\phi(x\,;b)\wedge\psi(x\,;c)\Big\}$
  
  contradicts the maximality of the cardinality of $\Gamma$.
\end{proof}

\begin{proposition}
  For every $f \in \RR^M$ and $\epsilon\in \RR^+$ there is a tuple $c$ of length $\le\Vert f\Vert/\epsilon$ satisfying Lemma~\ref{lem_uno}.
\end{proposition}

\begin{proof}

\def\medrel#1{\parbox[t]{5ex}{\hfil$\displaystyle #1$}}
\def\ceq#1#2#3{\parbox[t]{15ex}{$\displaystyle #1$}\medrel{#2}$\displaystyle  #3$}
  Let $\epsilon$ be given.
  Let $\gamma_f$ be the maximal integer such that, for some $c$, there is set $\Gamma$ of cardinality $\gamma_f$ containing pairwise inconsistent formulas $\psi(x\,;c)\in\{\wedge\}\Delta^\pm$ such that $\Vert f\cdot\psi(x\,;c)\Vert>\epsilon$.

  We claim that there is a $c$ witnessing  $\gamma_f$ such that $|c|\le\gamma_f$.
  Obviously $\gamma_f\le\Vert f\Vert/\epsilon$, hence the proposition follows from the claim.

  If $\gamma_f=0$, the claim holds trivially with as $c$ the empty tuple.
  Let $\gamma_f=k>0$ and assume the claim holds for every $f$ such that $\gamma_f<k$.

  Let $f'=f\cdot\phi(x\,;b)$ and $f''=f\cdot\neg\phi(x\,;b)$, where $b$ will be specified below.
  We check that $\gamma_{f'}+\gamma_{f''}\le\gamma_f$.
  In fact, pick $c',\Gamma'$ and  $c'',\Gamma''$ that witness $\gamma_{f'}$ and $\gamma_{f''}$ respectively.
  Then

  \ceq{\hfill\Gamma}{=}{\Big\{\phantom{\neg}\phi(x\,;b)\wedge\,\psi'(x\,;c')\;\ :\ \psi'(x\,;c')\in\Gamma'\Big\}\ \ \cup}

  \ceq{}{}{\Big\{\neg\phi(x\,;b)\wedge\psi''(x\,;c'')\ :\ \psi''(x\,;c'')\in\Gamma''\Big\}}

  is a set of cardinality $\gamma_{f'}+\gamma_{f''}$ containing pairwise inconsistent formulas of the form $\psi(x\,;c)\in\{\wedge\}\Delta^\pm$, where $c=c',c'',b$, such that $\Vert f\cdot\psi(x\,;c)\Vert>\epsilon$.

  The claim follows from $\gamma_{f'}+\gamma_{f''}\le\gamma_f$ and the induction hypothesis
  if we can find a $b$ such that $\gamma_{f'}, \gamma_{f''}>0$.

  Let $f$ and $c=c_1,\dots,c_n$ be such that $\Gamma$ has cardinality $\gamma_f$.
  Suppose also that $n$ is minimal.
  Let $d=c_1,\dots,c_{n-1}$.
  By the minimality of $n$, there is a complete formula $\psi(x\,;d)$ such that both $\psi(x\,;d)\wedge\phi(x\,;c_n)$ and $\psi(x\,;d)\wedge\neg\phi(x\,;c_n)$ are in $\Gamma$.
  Therefore if we set $b=c_n$, the sets 

  \ceq{\hfill\Gamma'}{=}{\Big\{\psi(x\,;d)\;\ :\ \psi(x\,;d)\wedge\phi(x\,;c_n)\in\Gamma\Big\}}

  \ceq{\hfill\Gamma''}{=}{\Big\{\psi(x\,;d)\;\ :\ \psi(x\,;d)\wedge\neg\phi(x\,;c_n)\in\Gamma\Big\}}

  witness that $\gamma_{f'}, \gamma_{f''}>0$.
\end{proof}


\begin{theorem}
  For every $f\in\RR^M$ and every $\epsilon>0$ there is a $c\in\big(M^z)^n$, where $n\le\Vert f\Vert/\epsilon$, such that for every $b\in M^z$ and every complete formula $\psi(x\,;c) \in \{\wedge\} \Delta^\pm$ one of the following occur


  \ceq{\hfill\big\Vert f \cdot \phi(x\,;b) \cdot \psi(x\,;c\big)\Vert}{\le}{\epsilon}
  \\[-.5ex]
  \ssf1.
  \\[-.5ex]
  \ceq{\hfill\big\Vert  f \cdot \neg\phi(x\,;b) \cdot \psi(x\,;c)\big\Vert}{\le}{\epsilon.}
\end{theorem}
 

If $f,g\in \U$ we write $g\ll_\Delta f$ if for every $\psi(x)$ that is conjunction of formulas in $\Delta$

\ceq{\hfill\sum_{\psi(x)}gx}{\lesslin}{\sum_{\psi(x)}fx}

\begin{lemma}
  If $g\ll_\Delta f$ then for every $\epsilon$ there is a simple function $h_\epsilon$ such that for every $\psi(x)$ that is conjunction of formulas in $\Delta$

  \ceq{\hfill\Big|\sum_{\psi(x)}gx-\sum_{\psi(x)}h_\epsilon x\cdot fx\Big|}{<}{\epsilon.}

\end{lemma}


for every formula $\epsilon\in\RR^+$ there is a $\delta\in\RR^+$ such that 


holds


$\phi(x\,;b)$



For every $\epsilon$ there is an function $k$ sich that

$\sum_{\phi(x\,;\,b)}gx-kfx<\epsilon$

%%%%%%%%%%%%%%%%%%%%%%%%%%%%
%%%%%%%%%%%%%%%%%%%%%%%%%%%%
%%%%%%%%%%%%%%%%%%%%%%%%%%%%
%%%%%%%%%%%%%%%%%%%%%%%%%%%%
%%%%%%%%%%%%%%%%%%%%%%%%%%%%
\section{Pseudofinite counting measure}



%%%%%%%%%%%%%%%%%%%%%%%%%%%%
%%%%%%%%%%%%%%%%%%%%%%%%%%%%
%%%%%%%%%%%%%%%%%%%%%%%%%%%%
%%%%%%%%%%%%%%%%%%%%%%%%%%%%
%%%%%%%%%%%%%%%%%%%%%%%%%%%%
\section{Pseudofinite counting measure}

\def\ceq#1#2#3{\parbox[t]{20ex}{$\displaystyle #1$}\medrel{#2}$\displaystyle  #3$}

Below, the theory $T_0$, the set of formulas $\Delta$, and the rank $m=R_\Delta$ are as defined the provious section.

Let $\D\subseteq\U$ be a definable set.
%
We define

\ceq{\hfill\Pr(\D)}
{=}
{\inf\Big\{\frac{m}{n}\ :\ m,n\in\NN\sm\{0\} \text{ such that } n\,\ns|\D|\le m\,\ns|\U|\Big\}},

where the infimum is taken in $\RR$.
%
It is immediate that $\Pr(\mbox{-})$ is a finite probability measure on the definable subsets of $\U$.
%
(By the Caratheodory Theorem it can be extended to a probability measure but this is not required here.)

We will write \emph{$\Aa\eqPr\D$} if $\Pr(\Aa\simdiff\D)=0$.

\begin{lemma}\label{lem_atoms}
  There are some $\{\wedge\}\pmDelta$-definable sets $\B_1,\dots,\B_n\subseteq\U$, where $n\le2^m$, such that for any $\{\wedge\}\pmDelta$-definable set $\Aa$, 
  
  \ceq{\hfill\Aa}{\eqPr}{\bigcup_{i\in I}\B_i,}

  per qualche $I\subseteq n$.
  %
  Also, we can require that all $\B_i$ have positive measure and that they form a partition $\U$.
\end{lemma}

\begin{proof}
  We prove a slightly more general claim.
  %
  Let $\D$ be a definable set of positive measure.
  %
  By induction on $R_\Delta(\D)=m$ we prove that there are $n\le 2^m$ many $\{\wedge\}\pmDelta$-definable sets $\B_i\subseteq\D$ such that for every $\{\wedge\}\pmDelta$-definable $\Aa\subseteq\D$ there is an $I\subseteq n$

  \ceq{\hfill\Aa}{\eqPr}{\bigcup_{i\in I}\B_i.}

  If  $R_\Delta(\D)=0$ then $\D$ is either disjoint of contained in every $\pmDelta$-definable set.
  So the claim holds trivially.

  Now, let $R_\Delta(\D)=m+1$.
  By the definition of binary rank, there is a $\Delta$-definable set $\B$ such that $R_\Delta(\D\cap\B)=R_\Delta(\D\cap\neg\B)=m$.
  Apply the induction hypothesis to $\D\cap\B$ if it has positive measure. 
  Do the same for $\D\cap\neg\B$.
  Note that at least one has positive measure.

  It is evident that the sets $\B_i$ obtained in the construction above are disjoint and have positive measure.
  Also, they cover $\U$ up to a set of measure $0$.
  Hence, replacing $\B_1$ with some $\B'_1\eqPr\B_1$ we obtain an actual cover of $\U$.
\end{proof}

\noindent\llap{\Large\color{red}\fontencoding{U}\fontfamily{futs}\selectfont\char 66\relax\ }%
Non sono sicuro questa qui sotto sia la corretta traduzione finita.

\begin{corollary}
  For every $\epsilon>0$ there is an $k$ such that for every finite $M\models T_0$ of cardinality larger than $k$, there is a partition of $M$, say $B_0,\dots,B_{n-1}\subseteq M$, where $n\le2^m$ such that $|B_i|>\epsilon\,|M|$ and for every $\{\wedge\}\pmDelta$-definable set $A\subseteq M$ there is an $I\subseteq n$ such that 

  \ceq{\hfill\bigg|A\simdiff\bigcup_{i\in I}B_i\bigg|}
  {<}
  {\epsilon\,|M|.}
\end{corollary}


%%%%%%%%%%%%%%%%%%%%%%%%%%%%%%%%%%%%%%%%
%%%%%%%%%%%%%%%%%%%%%%%%%%%%%%%%%%%%%%%%
%%%%%%%%%%%%%%%%%%%%%%%%%%%%%%%%%%%%%%%%
\section{Stable Szemer\'edi's regularity}

\def\ceq#1#2#3{\parbox[t]{15ex}{$\displaystyle #1$}\medrel{#2}$\displaystyle  #3$}
Below, the theory $T_0$, the set of formulas $\Delta$, and the rank $m=R_\Delta$ are as defined in Section~\ref{EH}.

\begin{lemma}
  There are two partitions of $\U$, say $\B_1,\dots,\B_n$, where $n\le 2^m$, and $\C_1,\dots,\C_{2^n}$, such that for all $i,j$ either $r(\B_i,\C_j)\eqPr\0$ or $r(\B_i,\C_j)\eqPr\B_i\times\C_j$.
\end{lemma}

\begin{proof}
  Let $\B_1,\dots,\B_n$ be the sets given by Lemma~\ref{lem_atoms}. For every $J\subseteq n$ let 
  
  \ceq{\hfill \C_J}{=}{\Big\{c\in\U\ :\ \ \bigcup_{i\in J}\B_i\;\eqPr r(\U,c)\Big\}}

  By Lemma~\ref{lem_atoms} if $i\in J$ then $r(\B_i,\C_J)\eqPr\B_i\times\C_J$, otherwise $r(\B_i,\C_J)\eqPr\0$.
\end{proof}

For $\Aa,\B\subseteq\U$, it is usual call \emph{density\/} the conditional probability

\ceq{\hfill\emph{$d(\Aa,\B)$}}
{=}
{\Pr\Big(r(\Aa,\B)\ \big|\ \Aa{\times}\B\Big)}

\ceq{}
{=}
{\frac{\Pr\big(r(\Aa,\B)\big)}{\Pr(\Aa{\times}\B)}}

This definition is given with the proviso that $\Aa$ and $\B$ have non zero probability.
%
The same notation is used for nonempty subsets $A,B\subseteq M$ of a finite model

\ceq{\hfill\emph{$d(A,B)$}}
{=}
{\frac{\big|r(A,B)\big|}{|A\times B|}}

\begin{corollary}
  There is a partition of $\U$ into sets of positive measure $\D_1,\dots,\D_n$, where $n\le2^{2^m+1}$ such that for all $i,j$ either $d(\D_i, \D_j)=0$ or $d(\D_i, \D_j)=1$.\QED
\end{corollary}


Finally, the the stable Szemer\'edi regularity lemma is obtained reasoning as in Fact~\ref{fact_application}.


\noindent\llap{\Large\color{red}\fontencoding{U}\fontfamily{futs}\selectfont\char 66\relax\ }%
%
Il corollario qui sotto sembra troppo forte.
Da quanto vedo in letteratura $n$ dovrebbe dipendere da $\epsilon$. 
Forse non \`e la traduzione al finito giusta.

\begin{corollary} 
  For every $\epsilon>0$ there is an $k$ such that for every finite $M\models T_0$ of cardinality larger than $k$, there is a partitions of $M$, say $D_1,\dots,D_n$, where $n\le2^{2^m+1}$ such that for all $i,j$ either $d(D_i,D_j)<\epsilon$ or $d(D_i,D_j)>1-\epsilon$.
  \QED
\end{corollary}


  % \ceq{\hfill\frac{|A\times B|}{|V_i\times V_j|}\cdot\bigg|\frac{|r(A,B)|}{|A\times B|} - \frac{|r(V_i,V_j)|}{|V_i\times V_j|}\bigg|}
  % {<}
  % {\epsilon}

  % \ceq{\hfill\Pr(A,B|V_i,V_j)\cdot\big|\Pr(r|A,B) - \Pr(r|V_i,V_j)\big|}
  % {<}
  % {\epsilon}


%%%%%%%%%%%%%%%%%%
%%%%%%%%%%%%%%%%%%
%%%%%%%%%%%%%%%%%%
%%%%%%%%%%%%%%%%%%
\clearpage\section{Bohhh}

\def\medrel#1{\parbox[t]{5ex}{\hfil$\displaystyle #1$}}
\def\ceq#1#2#3{\parbox[t]{25ex}{$\displaystyle #1$}\medrel{#2}$\displaystyle  #3$}


If $f,g\in \RR^M$ we write $g\ll_\Delta f$ if for every $\epsilon>0$ there is a $\delta>0$ such that

\ceq{\hfill\sum_{x\in X}|fx|<\delta}{\IMP}{\sum_{x\in X}|gx|<\epsilon}

for every $X$ in the Boolean algebra generated by $\Delta$.

\begin{fact}
  The following are equivalent
  \begin{itemize}
    \item[1.] $g\ll_\Delta f$
    \item[2.] for every $X$ in the Boolean algebra generated by $\Delta$
  \end{itemize}
  
\ceq{\hfill\sum_{x\in X}|fx|\sim 0}{\IMP}{\sum_{x\in X}|gx|\sim0}
\end{fact}

\begin{proof}
  \ssf1$\IMP$\ssf2 Asume \ssf1 and $\sum_{x\in X}|fx|\sim 0$. 
  Then $\sum_{x\in X}|gx|<\epsilon$ for every $\epsilon$ and therefore $\sum_{x\in X}|gx|\sim 0$.

  \ssf2$\IMP$\ssf1 Negate \ssf1. There is an $\epsilon$ Such that for every $\delta$ there is an $X\in\Delta_n$ such that
  
  \ceq{\hfill\sum_{x\in X}|fx|<\delta}{\wedge}{\sum_{x\in X}|gx|\ge\epsilon}



\end{proof}

\begin{theorem}
  If $g\ll_\Delta f$ then for every $\epsilon>0$ there is $\Delta$-simple function $k$ such that $\Vert g-k\,f\Vert\le\epsilon$
\end{theorem}

that is conjunction of formulas in $\Delta$

\begin{lemma}
  If $g\ll_\Delta f$ then for every $\epsilon$ there is a simple function $h_\epsilon$ such that for every $\psi(x)$ that is conjunction of formulas in $\Delta$

  \ceq{\hfill\Big|\sum_{\psi(x)}gx-\sum_{\psi(x)}h_\epsilon x\cdot fx\Big|}{<}{\epsilon.}

\end{lemma}


%%%%%%%%%%%%%%%%%%
%%%%%%%%%%%%%%%%%%
%%%%%%%%%%%%%%%%%%
%%%%%%%%%%%%%%%%%%
\clearpage\section{Bohhh}
Let $M$ range over the finite graphs.

Let $f:\NN\to\NN$ be an unbounded monotone function. 

For $m\in\NN\sm\{0\}$, we write $M\models R^f< m$ if there is no map 

\ceq{\hfill a\ :\ {}^{<m}2}{\to}{ M} 

\ceq{\hfill   s}{\mapsto}{a_s} 

such that for every $s\in {}^{m}2$ the finite type 

\ceq{\hfill \big\{\neg^{s_i}\,r(x\,;a_{s\restriction i})\, :\, i<m\big\}}
{}
{}

has more than $|M|^{1/f(|M|)}$ solutions.
%
We write $M\models R^f{=}\,m$ if $m$ is the minimal such that $M\models R^f{<}\, m+1$. 
%
If $S\subseteq M$ is a definable set, we write $M\models R^f(S)=m$ if all the solutions above are in $S$.

Let $T_m$ be the theory of graphs plus axioms that $R^f\le m$.

Let $k=f(\ns|\U|)$.
%
As $f$ is an unbounded monotone function $k>\NN$.

Among the large $L(\U)$-definable subsets of $\U$ we fix one, say \emph{$\S=\sigma(\U)$}, such that $R^f\big(\S\big)$ is minimal.

Let \emph{$\Delta$} $=\{r(a,x) :\, a\in\U\}$.

The following fact is immediate (note the assumtion $n>0$).

\begin{fact}
  Let $\S\subseteq\U$ and suppose that for some $\pmDelta$-definable set $\D$ both $\S\cap\D$ and $\S\cap\neg\D$ have rank $n>0$.
  %
  Then $R^f(\S)=n+1$.\QED
\end{fact}

\begin{fact}
  For every $\pmDelta$-definable set $\D\subseteq\U$, exactly one between $\S\cap\D$ and $\S\cap\neg\D$ is large.
\end{fact}

\begin{proof}
  As $\S$ is large, at least one between $\S\cap\D$ and $\S\cap\neg\D$ is large by Proposition~\ref{prop_+=max}.

  Firstly, assume $R^f(\S)=0$.
  %
  Then for every $a\in\U$ either $S\cap r(\U,a)$ or $\S\cap\neg r(\U,a)$ has cardinality $<\ns|\U|^{1/k}$, which implies that it is small.
  
  Now, assume $R^f(\S)=n>0$ and suppose for a contradiction that $\S\cap\D$ and $\S\cap\neg\D$ are both large.
  %
  By the minimality of $n$, both $\S\cap\D$ and $\S\cap\neg\D$ have rank $n$.
  %
  This contradicts the fact above.
\end{proof}

\begin{corollary}
  There is a (unique) complete $\pmDelta$-type $p(x)$ such that $\S\sm\phi(\U)$ is small for every $\phi(x)\in p$.\QED 
\end{corollary}

%%%%%%%%%%%%%%%%%%%%
%%%%%%%%%%%%%%%%%%%%
%%%%%%%%%%%%%%%%%%%%
\hrulefill
\def\ceq#1#2#3{\parbox[t]{15ex}{$\displaystyle #1$}\medrel{#2}$\displaystyle  #3$}

We say tha a type $p(x)\subseteq\pmDelta$ is \emph{p-isolated\/} by $\sigma(x)$ if  this is a p-large formula such that $\sigma(x)\wedge\neg\phi(x)$ is p-small for every $\phi(x)\in p$.  We may equivalently say that $\phi(x)$ is p-co-small in $\sigma(x)$. 

It is convenient to fix a model where to realze global types.
In the following $\<\V,\ns\ns\NN\>$ is a model such that $\<\U,\ns\NN\>\preceq\<\V,\ns\ns\NN\>$.
We say that $a\in\V$ is \emph{p-isolated\/} if some formula $\sigma(x)$ such that $a\models\sigma(x)$ p-isolates $\pmDelta\mbox{-}\tp(a)$.

For $a,b\in\V$, we write 

\ceq{\hfill b\mathop{\cnonfork\kern-.5ex^{'}}_\U a}
{\IFF}
{\phi(b,a)} holds for all $\phi(x,y)\in L'(\U)$ such that $\U\subseteq\phi(\V,a)$. 

 
This is the standard coheir-heir relation in the language $L'$.
%
Note that the standard numbers $\NN$ are definible in $L'$, hence they may occour in formulas of $L'(\U)$.



\begin{lemma}
  Let $a,b\in\V$ be such that $b\mathop{\cnonfork\kern-.5ex^{'}}_\U a$.
  %
  If $\pmDelta\mbox{-}\tp(a/\U)$ be p-isolated by $\sigma(x)$, then $\pmDelta\mbox{-}\tp(a/\U,b)$ is p-isolated by the same formula.
\end{lemma}
\begin{proof}
  Suppose not. 
  %
  Then there is a formula in $\pmDelta$ such that $\phi(b,a)$ and $\sigma(x)\wedge\neg\phi(b,x)$ is p-large. 
  %
  Then $\phi(b,a)\ \wedge\ \ns|\sigma(x)\wedge\neg\phi(b,x)|\,\ge\, \ns|x=x|^{1/n}$ for some $n\in\NN$.

  By the definition above $\neg\phi(c,a)\ \wedge\ \ns|\sigma(x)\wedge\neg\phi(c,x)|\,\ge\, \ns|x=x|^{1/n}$ for some $c\in\U$.
  %
  This contradicts the assumption that $\pmDelta\mbox{-}\tp(a/\U)$ is p-isolated.
\end{proof}


\begin{lemma}
  Let $a\in\V$ be such that $\pmDelta\mbox{-}\tp(a/\U)$ be p-isolated by $\sigma(x)$.
  %
  There is a p-long internal sequence $c=\<c_i:i<n\>$ such that either $r(c_i,c_j)$ for all $i<j<n$ or $\neg r(c_i,c_j)$ for all  $i<j<n$.  
\end{lemma}
\begin{proof}
  Let $a\equiv'_\U b$ be p-isolated by the same formula $\sigma(x)$.
  %
  Suppose that $b\mathop{\cnonfork\kern-.5ex^{'}}_\U a$ and that $r(b,a)$.
  %
  Then $r(b,x)$ is co-p-small in $\sigma(x)$.
  %
  As $a\equiv'_\U b$, for evey $n\in\NN$, the formula $\phi(y)=\ns|r(y,x)|_x<|x=x|^{1/n}$ is co-p-small in $\sigma(x)$. 

  Let $\B_n=\{b\in\V\ :\ \sigma(b) \textrm{ and } \ns|\neg r(b,x)|<\ns|x=x|^{1/n}\}$
\end{proof}
%\end{comment}
\chapter{Loeb samples}

\end{comment}




% Let $\U$, $\V$ and $\phi(x\,;z)$ be as in the notes. Question: is the following lemma true?

% \begin{lemma}
%   If $\phi(x\,;z)$ has \vc-dimension $n$ then there are $b\in\V$ and $A_0\subseteq\U$ such that for every $c\in\V$ 
  
%   \ceq{\hfill\phi(A_0,b)=\phi(A_0;c)}{\IMP}{\phi(\U,b)=\phi(\U;c)}

% moreover, $A_0$ is finite and its cardinality depends only on $n$.
% \end{lemma}

% Does Lemma~1 follows from Lemma~2 below? Probably not, but why not?

% \begin{lemma}
%   If $\phi(x\,;z)$ has \vc-dimension $n$ then for every finite $A\subseteq\U$ there are $b\in\V$ and $A_0\subseteq A$ such that for every $c\in\V$ 
  
%   \ceq{\hfill\phi(A_0,b)=\phi(A_0;c)}{\IMP}{\phi(A,b)=\phi(A;c)}

% moreover, the cardinality of $A_0$ depends only on $n$.
% \end{lemma}


\def\medrel#1{\parbox[t]{6ex}{\hfil$\displaystyle #1$}}
\def\ceq#1#2#3{\parbox[t]{23ex}{$\displaystyle #1$}\medrel{#2}$\displaystyle  #3$}

\section{Bootstrapping Loeb samples}

In this section we introduce hypefinite samples and prove in Lemma~\ref{lem_existence_sample} that all Keisler measures are generated by some Loeb sample.

Below \emph{$\U$\/} is a saturated model of a complete theory \emph{$T$\/} in the language \emph{$L$.}
We write  \emph{$\kappa$\/} for the cardinality of $\U$ and assume that $\kappa$ an inaccessible cardinal larger than $|L|$.

For every $n\in\omega$ define 

\ceq{\hfill\emph{$S_n$\/}}{=}{\Big\{s:\U^n\to\RR\ \ :\ \ s\,a=0\ \textrm{for all but finitely many }a\Big\}}

The elements of $S_n$ are called \emph{standard samples.}
These will interpreted as signed measures concentrated on a finite set.
%(We could restrict ourself to $\NN$-valued functions. 
%This would make the term \textit{sample\/} more appropiate, but would complicate the notation.)
We denote by \emph{$\bar\U$\/} the multi-sorted structure $\<\U,\RR,(S_n)_{n\in\omega}\>$.
Clearly, $|\bar\U|=\kappa$.
We call the first sort the \emph{home\/} sort; the second  one the \emph{real\/} sort and the remaining ones the \emph{sample sorts.}

The language of $\bar\U$ is denoted by \emph{$\bar L$.}
It contains $L$ and a symbol for every function $\RR^n\to\RR$. 
Moreover, for every formula in $\phi(x,z)\in L$ the language $\bar L$ contains a function symbol of sort

\ceq{\hfill S_{|x|}\times\U^{|z|}}{\to}{\RR}

that we interpret as the function that maps

\ceq{\ssf{1.}\hfill  (s,b)\ \ }{\mapsto}{\sum_{\phi(x,b)} s\,x.}

As the functions in $S_{|x|}$ are null almost everywhere, the sum in \ssf1 is well-defined.
We will use two informal but suggestive symbols for this function: \emph{$\sum_{\phi(x,b)} s\,x$\/} or \emph{$\mu_s\phi(x,b)$.}
When $\phi(x,b)$ is the formula $x=b$, we write \emph{$s\, b$.}

Let \emph{$\ns\bar\U=\<\ns\U,\ns\RR,(\nsS_n)_{n\in\omega}\>$\/} be some fixed elementary extension of $\bar\U$ that is saturated and has cardinality $>\kappa$.

\begin{comment}
There are two extensions of $\bar\U$ that are relevant in the following.
The larger one 
The second extension is a saturated model \emph{$\nso\bar\U$} such that $\bar\U\preceq\nso\bar\U\preceq\ns\bar\U$ and that $|\nso\bar\U|=\kappa$.
As $\kappa$ is inaccessible such model exists and we can assume that the domain of its home sort is $\U$.

\begin{remark}
  The model $\nso\bar\U$ plays the role of the moster model of $\bar T=\Th(\bar\U)$ while $\ns\bar\U$ is a model where all global types (i.e.\@ types over $\nso\bar\U$) are realized.
  In fact, for notational reasons we will only mention global types through their realizations in $\ns\bar\U$.
  Intuitively samples in $\ns\bar\U$ correspond to global Keisler measures while samples in $\nso\bar\U$ correspond to smooth measures.
  More on this topic in the next section.\QED
\end{remark}
\end{comment}

The elements of $\bigcup_{n\in\omega}\nsS_n$ are called \emph{(Loeb) samples.} 
The \emph{support\/} of $s\in\nsS_{|x|}$ is the definable (hyperfinite) set $\{a\in\ns\U^{|x|}:s\,a\neq0\}$ which we denote by \emph{$\supp s$.}

If $M\preceq\U$ we write \emph{$S_n{\restriction} M$\/} for the set of functions $s\in S_n$ such that $\supp s\subseteq M^n$.
We define \emph{$\bar M$\/} to be the structure $\<M,\RR,(S_n{\restriction} M)_{n\in\omega}\>$.

\begin{fact}\label{fact_elementarity}
  Let $M\preceq\U$ be $\omega$-saturated.
  Then $\bar M\preceq\bar\U$.
  In general, when $M$ is not saturated, for all sentences $\phi\in\bar L(\bar M)$ with no quantifiers of sample sort

  \ceq{\hfill\bar M\models\phi}{\IFF}{\bar\U\models\phi.}
\end{fact}

\begin{proof}
  We prove the second claim first.
  We can assume that the function $\mu_s\phi(x,z)$ only occurs in atomic formulas of the form $\mu_s\phi(x,z)=y$.

  Fix $s\in S_{|x|}{\restriction} M$.
  Let $a_1,\dots,a_n$ be an enumeration of $\supp s$ and define $\alpha_i=s\,a_i$.
  The formula $\mu_s\phi(x,z)=y$ is easily seen to be equivalent, both in $\bar M$ and in $\bar\U$, to the conjunction of the formulas
   
  \ceq{\hfill\bigwedge_{i=1}^n\neg^{\epsilon_i}\phi(a_i, z)} {\imp}{y=\sum_{i=1}^n[1-\epsilon_i]\cdot\alpha_i}

  as $\epsilon$ ranges over $^n2$.
  Hence, every sentence $\phi\in\bar L(\bar M)$ is equivalent to some sentence in $\psi\in L(M,\RR)$.
  As $\psi$ does not contain parameters nor quantifiers of sample sort, its truth in $\bar M$ and $\bar\U$ depends only on the structures $M,\RR$, respectvely $\U,\RR$.
  Then the equivalence above is a consequence of $M\preceq\U$.

  Now assume that $M$ is $\omega$-saturated.
  We need to prove that for every tuples $a,t$ in $\bar M$ of  home, respectively sample sort we have

  \ceq{\hfill\bar M\models\phi(a,t)}{\IFF}{\bar\U\models\phi(a,t)}\hfill for all $\phi(x,w)\in\bar L$.
  
  (There is no need to mention parameters in $\RR$ because they occur as constant in $\bar L$.) 
  
  Reason by induction on the syntax.
  The only interesting case concern the existential quantifier of samle sort, say $\E u$ where $u$ has the sort of $S_{|x|}$.
  If $\bar\U\models\E u\,\phi(u,a,t)$ then $\bar\U\models\phi(s,a,t)$ for some finite sample $s\in S_{|x|}$.
  Let $b_1,\dots,b_n\in\U^{|x|}$ enumerate the support of $s$. 
  By $\omega$-saturation, there are $c_1,\dots,c_n\in M^{|x|}$ such that $b_1,\dots,b_n\equiv_{a,\,\supp t}c_1,\dots,c_n$.
  By homogeneity, there is an $f\in\Aut(\U/a,\,\supp t)$ such that $fb_i=c_i$.
  Extend $f$ to an automorphism of $\bar\U$ by requiring that $f$ is the identity on $\RR$ and $f(s\,b)=(f\,s)(f\,b)$.
  Then $\bar\U\models\phi(fs,a,t)$, so $\bar M\models\phi(fs,a,t)$ follows by induction hypothesis.
\end{proof}

\begin{lemma}\label{lem_existence_sample}
  Let $\mu$ be finitely additive signed measures on $L_x(\U)$.
  Then there is $s\in\nsS_{|x|}$ such that 
  
  \ceq{\#\hfill\mu_s\phi(x)}{=}{\mu\phi(x)}\hfill for every $\phi(x)\in\Delta$.
\end{lemma}

\begin{proof}
  Let $u$ be a variable of sample sort.
  We claim that the type $p(u)$ defined below is finitely consistent 

  \ceq{\hfill p(u)}{=}{\Big\{\sum_{\phi(x)}u\,x=\mu\phi(x)\quad:\ \ \phi(x)\in L(\U)\Big\}}

  Let $\{\phi_1(x),\dots,\phi_n(x)\}\subseteq L(\U)$.
  It suffices to show that there is $s\in S_{|x|}$ such that

  \ceq{\ssf{1.}\hfill\sum_{\phi_i(x)}s\,x}{=}{\mu\phi_i(x)}\hfill for $i=1,\dots,n$.

  Without loss of generality we can assume that $\{\phi_1(x),\dots,\phi_n(x)\}$ is a Boolean algebra with atoms $\phi_1(x),\dots,\phi_k(x)$ for some $k\le n$.
  Pick some $a_1,\dots,a_k\in\U^{|x|}$ such that $a_i\models\phi_i(x)$.
  Pick $s\in S_{|x|}$ with support $\{a_1,\dots,a_k\}$ and such that
  
  \ceq{\hfill s\,a_i}{=}{\mu\phi_i(x)}\hfill  for $i=1,\dots,k$.

  Clearly \ssf{1} above is satisfied by the finite additivity of the measure.
\end{proof}

We say that $\mu_s$ is \emph{bounded\/} if there in an $\alpha\in\RR$ such that $|\mu_s|<\alpha$.

\begin{corollary}
  Let $s\in\nsS_{|x|}$ be such that $\mu_s$ is bounded.
  Then there is a $t\in\nsS_{|x|}$ such that $\mu_t=\st\big(\mu_s\big)$, where $\st$ denotes the standard part.\QED
\end{corollary}

%If $\alpha,\beta\in\ns\RR$ we write $\alpha\approx_\epsilon\beta$ if $|\alpha-\beta|\le\epsilon$ and $\alpha\approx\beta$ if $\alpha\approx_\epsilon\beta$ holds for every $\epsilon\in\RR^+$.

We write \emph{$\mu_s{\restriction} M=\mu_t{\restriction} M$\/} if $\mu_s\phi(x)=\mu_t\phi(x)$ for every $\phi(x)\in L(M)$.
The expression \emph{$\mu_s{\restriction} M\approx\mu_t{\restriction} M$\/} has a similar meaning.

\begin{conjecture}
   For every bounded $s,t\in\ns S_{|x|}$ the following are equivalent
  \begin{itemize}
    \item[1.] $s\equiv_Mt$;
    \item[2.] $\mu_s{\restriction} M\approx\mu_t{\restriction} M$.
  \end{itemize}
\end{conjecture}


%%%%%%%%%%%%%%%%%%%%%%%%%%%%%
%%%%%%%%%%%%%%%%%%%%%%%%%%%%%
%%%%%%%%%%%%%%%%%%%%%%%%%%%%%
%%%%%%%%%%%%%%%%%%%%%%%%%%%%%
%%%%%%%%%%%%%%%%%%%%%%%%%%%%%
%%%%%%%%%%%%%%%%%%%%%%%%%%%%%
\section{Smooth samples}

The notion of smooth measure has been introduced by Keisler in his seminal article.
It perfectly translates to samples.
One advantage of samples is that they allow to concretize the intuition behind smoothness.
Smooth samples are those that are realized (in a model smaller than $\ns\bar\U$).

We say that a non-negative sample $s\in\nsS_{|x|}$ is \emph{smooth\/} over $M$, if for every $\phi(x,z)\in L$, every $b\in\U^{|z|}$ and every $\epsilon\in\RR^+$ there is a formula $\psi(x)\in L(M)$ such that 

\ceq{\hfill\psi(x)}{\imp}{\phi(x,b)}

\ceq{\hfill\mu_s\psi(x)}{\approx_\epsilon}{\mu_s\phi(x,b).}

The following notion will be proven redundant but it is important to point it out.
If for a given $\phi(x,z)$ and $\epsilon$, finitely many  formulas $\psi(x)$ suffices for all $b\in\U^{|z|}$, we say that $s$ is \emph{uniformly smooth.}
Precisely, $s$ is uniformly smooth if for every $\phi(x,z)\in L$ and every $\epsilon\in\RR^+$ there are some finitely many formulas $\psi_1(x),\dots,\psi_n(x)\in L(M)$ such that for every $b\in\U^{|z|}$ there is an $i\in\{1,\dots,n\}$ such that

\ceq{\hfill\psi_i(x)}{\imp}{\phi(x,b)}

\ceq{\hfill\mu_s\psi_i(x)}{\approx_\epsilon}{\mu_s\phi(x,b).}

\begin{fact}
  Let $s\in\nsS_{|x|}$ be a non-negative bounded sample. 
  Then the following are equivalent
  \begin{itemize}
    \item[1.] $s$ is smooth over $M$;
    \item[2.] $s$ is uniformly smooth over $M$; 
    \item[3.] if $t\in\nsS_{|x|}$ is non-negative and $\mu_s{\restriction} M\approx\mu_t{\restriction} M$, then $\mu_s\approx\mu_t$.
  \end{itemize}
\end{fact}

\begin{proof}
  A compactness argument easily proves \ssf1$\IFF$\ssf2.
  
  \ssf1$\IMP$\ssf3\quad Fix arbitrarily $\epsilon\in\RR$ and $\phi(x)\in L(\U)$.
  We claim that $\mu_s\phi(x)\approx\mu_t\phi(x)$.
  Assume that $s$ is smooth and let $\psi\in L(M)$ be as above but with $\epsilon/3$ for $\epsilon$.
  Then $\mu_t\phi(x)\approx_{\epsilon/3}\mu_t\psi(x)\approx\mu_s\psi(x)\approx_{\epsilon/3}\mu_s\phi(x)$ and the claim follows.

  \ssf3$\IMP$\ssf1\quad Suppose $\phi(x,b)$ witness the failure of \ssf1. That is, $\mu_s\psi(x)\napprox_\epsilon\phi(x,b)$ for some $\epsilon\in\RR^+$ and all $\psi(x)\in L(M)$.
  Compactness and the fact that $\bar\U\preceq\ns\bar\U$ ensure the existence of $t$ of maximal (hypefinite) cardinality such that $t\equiv_Ms$ and

  \def\ceq#1#2#3{\parbox[t]{35ex}{$\displaystyle #1$}\medrel{#2}$\displaystyle  #3$}
  
  \ceq{\hfill\big(\A x\in\supp t\sm\supp s\big)\neg\phi(x,b)}{\wedge}{\big(\A x\in\supp s\sm\supp t\big)\,\phi(x,b);}

  \def\ceq#1#2#3{\parbox[t]{15ex}{$\displaystyle #1$}\medrel{#2}$\displaystyle  #3$}

  We claim that 

  \ceq{\hfill\mu_t\phi(x,b)}{\approx}{\ \,\sup\big\{\mu_t\psi(x)\ :\ \psi(x)\imp\phi(x)\big\}.}

  This contraddicts \ssf3. 
\end{proof}

\begin{fact}
  Let $s\in\nsS_{|x|}$ be a non-negative bounded sample. 
  Then the following are equivalent
  \begin{itemize}
    \item[1.] $s$ is smooth over $M$;
    \item[2.] every $M$-saturated extension of $\bar M$ contains a sample $t$ such that $\mu_t\approx\mu_s$.
  \end{itemize}

  \begin{proof}
    \ssf1$\IMP$\ssf2\quad As $s$ is bounded, there is a type $p(u)\subseteq\bar L(M)$ that says $\mu_u{\restriction}M\approx\mu_s{\restriction}M$.
    Let $t\models p(u)$.
    By \ssf3 of the fact above $\mu_t\approx\mu_s$.

    \ssf2$\IMP$\ssf1\quad 
  \end{proof}
\end{fact}

%%%%%%%%%%%%%%%%%%%%%%%%%%%%%
%%%%%%%%%%%%%%%%%%%%%%%%%%%%%
%%%%%%%%%%%%%%%%%%%%%%%%%%%%%
%%%%%%%%%%%%%%%%%%%%%%%%%%%%%
%%%%%%%%%%%%%%%%%%%%%%%%%%%%%
%%%%%%%%%%%%%%%%%%%%%%%%%%%%%
\section{Pseudofinite samples}


In this section we define pseudofiniteness, a very strong form of  finite satisfiability. 
Note that there are a few distinct notions of finite satisfiabilty that apply to this context, however 
pseudofiniteness is one of the most natural.
(It might be stronger than the notion of finite satisfiability that applies to Keisler measures.)

We say that a Loeb sample $s\in\nsS_{|x|}$ is \emph{weakly pseudofinite\/} over $M$ if

\ceq{\ssf1.\hfill \phi(s)}{\IMP}{\phi(t)\textrm{ for some }t\in\bar M}\hfill for every $\phi(u)\in\bar L(M)$.

% or, equivalently,

% \ceq{\ssf2.\hfill\phi(\bar M)=S_{|x|}{\restriction}M}{\IMP}{\phi(s)}\hfill   for every $\phi(u)\in\bar L(\bar M)$

By Fact~\ref{fact_elementarity}, every sample is weakly pseudofinite over any $\omega$-saturated model.

If \ssf1 holds for every $\phi(u)\in\bar L(\U)$ then we say that $s$ is \emph{pseudofinite.}
By the standard argument of existence of global coheirs, for every weakly pseudofinite $s\in\nsS_{|x|}$ there is a pseudofinite $s'\in\nsS_{|x|}$ such that $s'\equiv_{M}s$.

\begin{comment}
The following example should justify the terminology.

\begin{example} 
  We prove the following claim.
  Let $L$ be the language of graphs.
  Let $T$ be the theory of the random graph.
  Fix some $M\preceq\U$ and let $s\in\nsS_1$ be a weakly pseudofinite sample over $M$.
  Then $\supp s$ is a pseudofinite graph.

  Recall the definition of pseudofinite graph.
  Let $T_{\rm fg}$ the set of sentences in $L$ that hold in every finite graph. A \textit{pseudofinite graph\/} is any structure that models $T_{\rm fg}$.

  If $\phi\in L$ is a sentence, we denote by $\bar\phi(u)$ the formula obtained by replacing in $\phi$ the quantifiers $\E x$ and $\A x$ with their bounded form: $\E x\in\supp u$, respectively $\A x\in\supp u$.
  Then for all $s\in\nsS_1$, we have that $\bar\phi(s)$ if and only if $\supp s\models\phi$.

  To prove the claim, suppose that $\supp s\notmodels\phi$.
  From the paragraph above we obtain that $\neg\bar\phi(s)$. 
  Hence, by pseudofiniteness, $\neg\bar\phi(t)$ holds for some $t\in S_1{\restriction}M$.
  As $\supp t$ is a finite graph, $\phi\notin T_{\rm fg}$.\QED
\end{example}
\end{comment}

To illustrate the notion of pseudofiniteness we prove the following simple fact.

\begin{fact}
  Let $s\in\nsS_{|x|}$ be pseudofinite and let $M\preceq\U$.
  Then every formula $\phi(x)\in L(\U)$ such that $\phi(M)\subseteq\supp s$ is algebraic. 
  In particular, if $\supp s$ is definable by a formula in $L(\U)$ then it is finite.
\end{fact}

\begin{proof}
  If $\phi(M)\subseteq\supp s$ the formula $\phi(u)$ below is satisfied by $s$
  
  \ceq{\hfill\A x \Big[\phi(x)}{\imp} {x\in\supp u\Big]}

  By pseudofiniteness this formula is satisfied in $\bar M$ hence $\phi(M)$ is finite.
\end{proof}


% \begin{fact}
%   For every $s\in\nsS_{|x|}$ there is $t\in\nsS_{|x|}$ strongly pseudofinite over $M$ such that $\mu_t{\restriction}M=\st\big(\mu_s{\restriction} M\big)$.
% \end{fact}

\begin{question} For what theories does the following hold?
  For every pseudofinite $s\in\nsS_{|x|}$ there is an $s'\equiv_Ms$ such that for every $\phi(u)\in\bar L(\U)$

\ceq{\ssf2.\hfill \phi(s')}{\IMP}{\textrm{there is a finite }A\subseteq M\textrm{ such that }\phi(s'\cdot \Indicator_A).}

A similar question may be asked for $s$ weakly pseudofinite and $\phi(u)\in\bar L(M)$.\QED
\end{question}


%%%%%%%%%%%%%%%%%%%%%%%%%%%%%
%%%%%%%%%%%%%%%%%%%%%%%%%%%%%
%%%%%%%%%%%%%%%%%%%%%%%%%%%%%
%%%%%%%%%%%%%%%%%%%%%%%%%%%%%
%%%%%%%%%%%%%%%%%%%%%%%%%%%%%
%%%%%%%%%%%%%%%%%%%%%%%%%%%%%
\section{Definable samples}

The following notion is apparently unrelated to the homonimous notion for Keisler measures, still it is quite natural.

We say that the sample $s\in\nsS_{|x|}$ is \emph{definable\/} over $M$ if for every $\phi(u,x)\in\bar L(\U)$ there is a formula $\psi(u,x)\in\bar L(M)$ such that $\psi(t,\U)=\phi(s,\U)$ for some $t\in\bar M$.

\begin{fact}[???]
  If $s\in\nsS_{|x|}$ is smooth over $M$ then it is definable over $M$.
\end{fact}




%%%%%%%%%%%%%%%%%%%%%%%%%%%%%
%%%%%%%%%%%%%%%%%%%%%%%%%%%%%
%%%%%%%%%%%%%%%%%%%%%%%%%%%%%
%%%%%%%%%%%%%%%%%%%%%%%%%%%%%
%%%%%%%%%%%%%%%%%%%%%%%%%%%%%
%%%%%%%%%%%%%%%%%%%%%%%%%%%%%
\section{Generically stable samples}

A sample $s\in\nsS_{|x|}$ is called \emph{generically stable\/} over $M$ if it is both peudofinite and definable over $M$.

\begin{fact}[???]
  Let $s\in\nsS_{|x|}$ be bounded and generically stable.
  Then for every formula $\phi(x,z)\in L(M)$ and every $\epsilon\in\RR^+$ there is a $t\in\bar M$ such that $\mu_t\phi(x,b)\approx_\epsilon\mu_s\phi(x,b)$ for every $b\in\U^{|z|}$.
\end{fact}


\begin{proof}
  For every $\alpha\in\RR$ let $\psi_\alpha(x)\in L(M)$ be such that $\psi_\alpha(\U)=\{b:\mu_s\phi(x,b)\approx_{\epsilon/2}\alpha\}$ there a $t\in\bar M$ such that  $\mu_t\phi(x,b)\approx_{\epsilon/2}\alpha$.
\end{proof}

%%%%%%%%%%%%%%%%%%%%%%%%%%%%%
%%%%%%%%%%%%%%%%%%%%%%%%%%%%%
%%%%%%%%%%%%%%%%%%%%%%%%%%%%%
%%%%%%%%%%%%%%%%%%%%%%%%%%%%%
%%%%%%%%%%%%%%%%%%%%%%%%%%%%%
%%%%%%%%%%%%%%%%%%%%%%%%%%%%%
\section{Invariant samples I}

We introduce the notions of \textit{invariant\/} and \textit{finitely satisfiable\/} samples.
There are two sensible variants of these notions.
Here we consider the most stringent variant, the less stringent one is discussed in the following section.

We write $\Aut(\ns\U/A,\{\U\})$ for the set of automorphisms of $\ns\U$ that fix $A$ pointwise and $\U$ setwise. 
Note that every automorphism $f\in\Aut(\ns\U)$ has a canonical extension to an automorphism in $\Aut(\ns\bar\U)$, which we denote by the same symbol $f$.
Namely, this is the extension that is the identity on $\ns\RR$ and that maps $s\in\nsS_n$ to the unique $fs\in\nsS_n$ such that $(fs)\,(fa)=f(s\,a)$.

Given $s\in\nsS_{|x|}$, we say that $\mu_s$ is \emph{invariant\/} over $A$ if for every $f\in\Aut(\ns\U/A,\{\U\})$

\ceq{\hfill\mu_s}{=}{\mu_{f\,s}}

Note that this is equivalent to requiring that 

\ceq{\hfill\mu_s\phi(x,b)}{=}{\mu_s\phi(x,fb)}\hfill for every $\phi(x,z)\in L$ and $b\in\U^{|z|}$.


\begin{definition}\label{def_fin_sat}
  We say that $s$ is \emph{finitely satisfiable\/} in $M$ if for every $\phi(x)\in L(\U)$

  \ceq{\hfill \phi(\supp s)\neq\0}{\IMP}{\phi(M)\neq\0.}\QED
\end{definition}

The following lemma shows that the finite satisfiability of a sample corresponds (in a sense) to the finite satisfiability of the associated Keisler measure.

\begin{lemma}
  Let $\mu$ be as in Lemma~\ref{lem_existence_sample} with $\Delta=L(\U)$.
  Assume that

  \ceq{\hfill \mu\phi(x)\neq0}{\IMP}{\phi(M)\neq\0}\hfill for every $\phi(x)\in\Delta$.
  
  Then there is $s\in\nsS_{|x|}$ that is finitely satisfied in $M$ and 
  
  \ceq{\hfill\mu_s\phi(x)}{=}{\mu\phi(x)}\hfill for every $\phi(x)\in\Delta$. 
\end{lemma}

\begin{proof}
  Let $p(u)$ be as in the proof of Lemma~\ref{lem_existence_sample}.
  Define 

  \ceq{\hfill q(u)}{=}{\Big\{\A x\,[\phi(x)\imp ux=0]\ \ :\ \ \phi(x)\in\Delta,\ \phi(M)=\0\Big\}}

  We need to show that $p(u)\cup q(u)$ is finitely consistent.
  Apply the same reasoning as in the proof of Lemma~\ref{lem_existence_sample}.% and note that Definition~\ref{def_independence} is automatically saisfied.
\end{proof}

\begin{fact}
  Every sample $s\in\nsS_{|x|}$ that is finitely satisfiable in $M$ is invariant over $M$.  
\end{fact}

\begin{proof}
  If $s$ is not $M$-invariant then for some $f\in\Aut(\ns\U/M,\{\U\})$, some $\phi(x,z)\in L$ and $b\in\U^{|z|}$ 

  \ceq{\hfill \mu_s\phi(x,b)}{\neq}{\mu_s\phi(x,fb)}
  
  In particular

  \ceq{\hfill 0}{\neq}{\mu_s\Big(\phi(x,b)\niff\phi(x,fb)\Big)}

  Then there is $a\in\supp s$ such that $\phi(a,b)\niff\phi(a,fb)$.
  Hence, from the finite satisfiability of $s$, we obtain that $\phi(M,b)\neq\phi(M,fb)$.
  This contradicts the $M$-invarance of $\mu_s$.
\end{proof}

%%%%%%%%%%%%%%%%%%%%%%%%%%%%%
%%%%%%%%%%%%%%%%%%%%%%%%%%%%%
%%%%%%%%%%%%%%%%%%%%%%%%%%%%%
%%%%%%%%%%%%%%%%%%%%%%%%%%%%%
%%%%%%%%%%%%%%%%%%%%%%%%%%%%%
%%%%%%%%%%%%%%%%%%%%%%%%%%%%%
\section{Invariant samples II}

The exposition is parallel to that of the previous section with no significant differences.

We write $\Aut(\ns\bar\U/A,\{\U\})$ for the set of automorphisms of $\ns\bar\U$ that fix $A$ pointwise and $\U$ setwise. 
Given $s\in\nsS_{|x|}$, we say that $\mu_s$ is \emph{weakly invariant\/} over $A$ if for every $f\in\Aut(\ns\bar\U/A,\{\U\})$

\ceq{\hfill\mu_s}{\approx}{\mu_{f\,s}}

Note that this is equivalent to requiring that 

\ceq{\hfill\mu_s\phi(x,b)}{\approx}{\mu_s\phi(x,fb)}\hfill for every $\phi(x,z)\in L$ and $b\in\U^{|z|}$.


\begin{definition}\label{def_wfin_sat}
  We say that $s$ is \emph{weakly finitely satisfiable\/} in $M$ if for every $\phi(x)\in L(\U)$

  \ceq{\hfill \mu_s\phi(x)\napprox 0}{\IMP}{\phi(M)\neq\0.}\QED
\end{definition}



\begin{fact}
  Every sample $s\in\nsS_{|x|}$ that is weakly finitely satisfiable in $M$ is weakly invariant over $M$.  
\end{fact}

\begin{proof}
  If $s$ is not weakly $M$-invariant then for some $f\in\Aut(\ns\bar\U/A,\{\U\})$, some $\phi(x,z)\in L$ and $b\in\U^{|z|}$ 

  \ceq{\hfill \mu_s\phi(x,b)}{\napprox}{\mu_s\phi(x,fb)}
  
  In particular

  \ceq{\hfill 0}{\napprox}{\mu_s\Big(\phi(x,b)\niff\phi(x,fb)\Big)}

  Then, by the finite satisfiability of $s$, we obtain that $\phi(M,b)\neq\phi(M,fb)$.
  This contradicts the $M$-invariance of $\mu_s$.
\end{proof}




%%%%%%%%%%%%%%%%%%%%%%%%%%%%%
%%%%%%%%%%%%%%%%%%%%%%%%%%%%%
%%%%%%%%%%%%%%%%%%%%%%%%%%%%%
%%%%%%%%%%%%%%%%%%%%%%%%%%%%%
%%%%%%%%%%%%%%%%%%%%%%%%%%%%%
%%%%%%%%%%%%%%%%%%%%%%%%%%%%%
\end{document}

%%%%%%%%%%%%%%%%%%%%%%%%%%%%%
%%%%%%%%%%%%%%%%%%%%%%%%%%%%%
%%%%%%%%%%%%%%%%%%%%%%%%%%%%%
%%%%%%%%%%%%%%%%%%%%%%%%%%%%%
%%%%%%%%%%%%%%%%%%%%%%%%%%%%%
%%%%%%%%%%%%%%%%%%%%%%%%%%%%%
\section{The Radon-Nikodyn theorem}

\def\medrel#1{\parbox[t]{5ex}{\hfil$\displaystyle #1$}}
\def\ceq#1#2#3{\parbox[t]{25ex}{$\displaystyle #1$}\medrel{#2}
$\displaystyle  #3$}

\begin{definition}
  Let $\mu,\nu$ be finitely additive measures on $\Delta$.
  We write \emph{$\nu\ll_\Delta\mu$\/} if for every $\epsilon\in\RR^+$ there is a $\delta\in\RR^+$ such that 

  \ceq{\ssf1.\hfill|\mu|\phi(x)\le\delta}{\imp}{|\nu|\phi(x)\le\epsilon}
  
  holds for every $\phi(x)\in\Delta$.
  We write \emph{$\nu\ll\mu$\/} if \ssf1 holds for all $\phi(x)\in \bar L(\bar\U)$.\QED
\end{definition}





\begin{lemma}
  Let $\mu,\nu$ be finitely additive signed measures on $\Delta$, a Boolean algebra of small cardinality.
  Assume also that $\mu,\nu$ are bounded,   $\nu\ll_\Delta\mu$ and $\mu\ge0$.
  Then there are $f\in\F_{|x|}$ and, for every $\epsilon\in\RR^+$ a simple function $g_\epsilon\in\F_{|x|}$ such that 
  
  \ceq{\hfill\mu_f\phi(x)}{\approx}{\mu\phi(x)}
  
  
  \ceq{\hfill\mu_{g_\epsilon\cdot f}\phi(x)}{\;\approx_\epsilon}{\nu\phi(x)}\hfill for every $\phi(x)\in\Delta$.
     
\end{lemma}

\begin{proof}
\def\ceq#1#2#3{\parbox[t]{15ex}{$\displaystyle #1$}\medrel{#2}$\displaystyle  #3$}
  As we can work separatey with $\nu^+$ and $\nu^-$, we can assume $\nu\ge0$.
  Let $\delta_\epsilon$ be a function that witnesses $\nu\ll_\Delta\mu$.
  Without loss of generality, we assume that $\delta_\epsilon\to0$ as $\epsilon\to0$.
  
  For every $\epsilon$ fix a maximal partition $\theta_{\epsilon,1}(x),\dots,\theta_{\epsilon,n_\epsilon}$ such that $\mu\,\theta_{\epsilon,i}(x)>\delta_\epsilon$.
  Define, for $a\models\theta_{\epsilon,i}(x)$ 

  \ceq{\hfill g_\epsilon(a)}{=}{\nu\,\theta_{\epsilon,i}(x)/\mu\,\theta_{\epsilon,i}(x)}
  
  Let $u$ be variable of sort $\F_{|x|}$.
  We claim that the type $p(u)$ defined below is finitely consistent. 

  \ceq{\hfill p(u)}{=}{\Big\{\sum_{\phi(x)} ux\approx_{\delta_\epsilon}\mu\phi(x)\quad:\ \ \phi(x)\in\Delta\Big\}}

  \ceq{\hfill }{\cup}{\Big\{\sum_{\phi(x)}g_\epsilon x\cdot ux\approx_\epsilon\nu\phi(x)\quad:\ \ \phi(x)\in\Delta,\ \epsilon\in\RR^+ \Big\}}

  Let $\{\phi_1(x),\dots,\phi_n(x)\}\subseteq\Delta$.
  We need to show that there are $f, g\in\F_{|x|}$ such that
  
  \def\ceq#1#2#3{\parbox[t]{33ex}{$\displaystyle #1$}\medrel{#2}$\displaystyle  #3$}

  \ceq{\ssf{1.}\hfill\sum_{\phi_i(x)}fx}{=}{\mu\phi_i(x)}\hfill for $i=1,\dots,n$;

  \ceq{\ssf{2.}\hfill\sum_{\phi_i(x)}gx}{=}{\nu\phi_i(x)}\hfill for $i=1,\dots,n$;

  and 
  
  \ceq{\ssf{3.}\hfill \A a\subseteq\supp u\,\Big[\sum_{x\in a}f x\le\delta(\epsilon)}{\imp}{\sum_{x\in a}gx\le\epsilon\Big]}\hfill for every $\epsilon\in\RR^+$.

  Without loss of generality we can assume that $\{\phi_1(x),\dots,\phi_n(x)\}$ is a Boolean algebra with atoms $\phi_1(x),\dots,\phi_k(x)$.
  Pick some $a_1,\dots,a_k\in\U^{|x|}$ such that $a_i\models\phi(x)$.
  Pick $f, g$ with support $\{a_1,\dots,a_k\}$ and such that
  
  \ceq{\hfill f(a_i)}{=}{\mu\phi_i(x)}\hfill  for $i=1,\dots,k$;

  \ceq{\hfill g(a_i)}{=}{\nu\phi_i(x)}\hfill  for $i=1,\dots,k$.

  Clearly \ssf{1} and \ssf{2} above are satisfied by the finite additivity of the measure.
  As for \ssf{3}, it suffices to verify that

  \ceq{\hfill \sum_{j\in J}fa_i\le\delta(\epsilon)}{\imp}{\sum_{j\in J}ga_j\le\epsilon}\hfill  for every $J\subseteq\{1,\dots,k\}$.

  By the definition of $f,g$ this is equivalent to

  \ceq{\hfill \mu{\phi_i(x)}\le\delta(\epsilon)}{\imp}{{\nu\phi_i(x)}\le\epsilon}\hfill  for some $i=1,\dots,n$

  which is holds because $\nu\ll_\Delta\mu$.
\end{proof}






The \emph{norm\/} of $f\in\nsS_{|x|}$ is defined as

\ceq{\hfill \emph{$\Vert f\Vert$}}{=}{\sum_{x=x} |fx|}.\QED


The (internal) cardinality of the supprt is denoted by \emph{$\vert \supp f\vert$.} 
In general, $\vert \supp f\vert$ is a hyperfinite integer.
If $a,b\in\nsS_{|x|}$ are $\{0,1\}$-valued we confused them with their support.
E.g., we write $a\subseteq b$ for $\supp a\subseteq\supp b$.




% \begin{notation}
% \end{notation}

  
% For $a,b\in\ns\RR$, we write \emph{$a\sim b$\/} if $|a-b|$ is infinitesimal, i.e.\@ $<\epsilon$ for every $\epsilon\in\RR^+$.
%   The following lemma is not required. 

%   \begin{lemma}\label{lem_3}
%     The following are equivalent for any $f,g\in\F_{|x|}$
%     \begin{itemize}
%       \item[1.] $g\ll f$
%       \item[2.] $|f a|\sim 0 \IMP |ga|\sim 0$ for all $a\in\U^{|x|}$.
%     \end{itemize}
%   \end{lemma}
  
%   \begin{proof}
%     Left as an exercise for the reader.
%   \end{proof}
  
% \begin{fact}
%   The following are equivalent for $f,g\in\F_{|x|}$ such that $\Vert g\Vert_\infty$ is finite.
%   \begin{itemize}
%     \item[1.] $g\ll f$
%     \item[2.] for every $\epsilon\in\RR^+$ there is a $\lambda\in\RR^+$ such that $\vert gx\vert\le\max\{\lambda\vert fx\vert,\,\epsilon\}$.
%   \end{itemize}
% \end{fact}

% \begin{proof}
%   \ssf1$\IMP$\ssf2 Fix $\epsilon$ and let $\delta$ be as given by $g\ll f$.
%   Let $\lambda = \Vert g\Vert_\infty / \delta$.
%   Let $a\in\U^{|x|}$ be given.
%   If $\vert fa\vert<\delta$ then $\vert ga\vert\le\epsilon$, hence \ssf2 holds.
%   Otherwise $\lambda\vert fa\vert\ge\lambda\delta\ge\Vert g\Vert_\infty$, which also implies \ssf2.

%   \ssf2$\IMP$\ssf1
%   is proved similarly --it does not need $\Vert g\Vert_\infty$ to be finite.
% \end{proof}




% \begin{definition}
%   We say that $f,g\in\F_{|x|}$ are \emph{$\Delta$-equivalent} if

%   \ceq{\hfill \sum_{\phi(x)} gx}{=}{\sum_{\phi(x)} fx}
  
%   holds for every $\phi(x)\in\Delta$.\QED
% \end{definition}


\begin{fact}
  Assume $\Delta$ is a Boolean algebra of small cardinality.
  Let $f,g\in\F_{|x|}$ be such that $f\ge0$ and $\Vert g\Vert_\infty$ is finite.
  Then for every $\epsilon\in\RR^+$ there are $h\in\F_{|x|}$ and $\lambda\in\RR^+$ such that $\vert h\vert\le \lambda\,f$ and $\Vert g - h\Vert\le\epsilon$.
\end{fact}

\begin{proof}
  Given $\epsilon$, let $\delta$ be as given by the assumption $g\ll l$.
  Let $n\in\NN$ be such that $|g|\le n$.
\end{proof}


\begin{definition}
  For $g\in \F_{|x|}$ we say that $g$ is \emph{$\Delta$-simple\/} if it is (finite) linear combination of indicator functions of sets in $\Delta$.\QED
\end{definition}
   

\begin{theorem}
  Assume $\Delta$ is a Boolean algebra of small cardinality.
  Let $f,g\in\F_{|x|}$ and assume $f\ge0$.
  If $g\ll_\Delta f$ than for every $\epsilon\in\RR^+$ there is a $\Delta$-simple $h$ such that $\Vert g-h\cdot f\Vert\le\epsilon$.
\end{theorem}


% \begin{definition}
%   Let $\phi(x,z)\in \bar L$ be fixed.
%   We say that $f,g\in\F_{|x|}$ are \emph{equivalent} if

%   \ceq{\hfill \sum_{\phi(x,b)} gx}{=}{\sum_{\phi(x,b)} fx}

%   for all $b\in\U^{|z|}$.
%   The {support} of $f\in\F_{|x|}$ is the set \emph{$\supp f$} = $\{a\in\U^{|x|}:fa\neq0\}$.
%   We say that $f$ has \emph{minimal support} if there is no equivalent $g$ such that $\supp g\subset\supp f$.
  
%   We say that $f$ has bounded variation if $\sum_{x=x}|fx|<n$ for some finite (standard) $n$.\QED
% \end{definition}


% We say that $f$ has \emph{$\Delta$-minimal support} if it is not equivalent to any $g$ with support properly contained in $\supp f$.%

% When the notion of $\Delta$-equivalence is first-order we have interesting consequences. 

% \begin{lemma}
%   Suppose $\Delta=\{\phi(x,b):b\in\U^{|z|}\}$.
%   Then every $g$ is equivalent to some $f$ with minimal support.
%   If $f$ has $\Delta$-minimal support, then $\phi(a,\U)\neq\phi(a',\U)$ for any two distinct $a,a'\in\supp f$.\QED
%   % For every $\epsilon$ there is an $n$ such that for every $f$ with finite variation there is a $g$ such that $\supp g$ has carditality $\le n$ and for every $b\in\U^{|z|}$

%   % \ceq{\hfill \sum_{\phi(x,b)} \vert gx- fx\vert}{<}{\epsilon}
% \end{lemma}

% \begin{proof}
%   This is true in $\<\U,\RR,F_1, F_2,\dots\>$ and, by elementarity, in $\bar\U$.
% \end{proof}

% \begin{lemma}
%   Suppose $\Delta=\{\phi(x,b):b\in\U^{|z|}\}$.
%   Then following are equivalent
%   \begin{itemize}
%     \item[1.] $g\ll_\Delta f$\smallskip
%     \item[2.]
%     \noindent\hskip-\labelwidth\hskip-\labelsep
%     \ceq{\hfill\sum_{\phi(x,b)}|fx|\sim0}{\IMP}{\sum_{\phi(x,b)}|gx|\sim0}\hfill for all $b\in\U^{|z|}$
%     \item[3.] for every $\epsilon\in\RR^+$ there is a $\delta\in\RR^+$ such that 
    
%     \noindent\hskip-\labelwidth\hskip-\labelsep
%     \ceq{\hfill\A z\ \bigg[\sum_{\phi(x,z)}|fx|<\delta}{\imp}{\sum_{\phi(x,z)}|gx|<\epsilon\bigg].}
%   \end{itemize}
% \end{lemma}

% \begin{proof}
%   Same proof as Lemma~\ref{lem_3}.
% \end{proof}

% \end{document}
% \begin{lemma}
%   If $g,f$ have bounded variation and $g\ll_\Delta f$.
%   Then for every $\epsilon\in\RR^+$ there are $k\in\NN$ and $h\in\F_{|x|}$ such that $k\cdot f\le h\le k\cdot f$ and $\Vert g-h\Vert<\epsilon$. 
% \end{lemma}

% Let ${\sf M}_n(A)$ be the set of finitely additive signed measures on the $A$-definable subsets of $\U^n$.
% We require that all $\mu\in{\sf M}_n(A)$ have bounded variation, i.e.\@ $\Vert\mu\Vert=|\mu|(x{=}x)$ is finite.

% For $\nu,\mu\in{\sf M}_n(A)$ we write $\nu\ll\mu$ if for some function

% \ceq{\hfill \delta\ :\ \big[0,\ \Vert\nu\Vert\big]}{\to}{\big[0,\ \Vert\mu\Vert\big]}\hfill (intervals of $\RR$)

% \ceq{\hfill \epsilon\ \ }{\mapsto}{\delta_\epsilon,}

% where $\delta_0=0$ and $\delta_{\Vert\nu\Vert}=\Vert\mu\Vert$, 
% we have that for all $\epsilon\ge0$ and all $\phi(x)\in L(A)$

% \ceq{\hfill|\mu|\,\phi(x) \le \delta_\epsilon}{\imp}{|\nu|\phi(x)\le\epsilon.}

% \begin{lemma}
%   For every $\mu\in{\sf M}_n(A)$ there is $f_\mu\in\F_{|x|}$ such that for every $\phi(x)\in L(A)$

%   \ceq{\hfill \sum_{\phi(x)}f_\mu x}{=}{\mu\,\phi(x).}

%   Moreover if $\nu\in{\sf M}_n(A)$ is such that $\nu\ll\mu$ then for every $\epsilon\in\RR\sm\RR^-$

%   \ceq{\hfill \A x \ \Big[\,f_\mu x \le \delta_\epsilon}{\imp}{f_\nu x\le\epsilon\,\Big].}
% \end{lemma}

% \begin{lemma}
%   For every $\mu,\nu\in{\sf M}_{|x|}(A)$, with $0\le\mu$, the following are equivalent
%   \begin{itemize}
%     \item[1.] $\nu\ll\mu$
%     \item[2.] for every $\epsilon\in\RR^+$ there is an $f\in\F_{|x|}$ such that $| f_\nu-f|<\epsilon$ and $|f|\le k\cdot f_\mu$ for some $k\in\RR$;
%     \item[3.] for every $\epsilon\in\RR^+$ there is a simple $f\in\F_{|x|}$ such that $\Vert f_\nu-f\cdot f_\mu\Vert<\epsilon$.
%   \end{itemize}
% \end{lemma}

% \begin{proof}
%   We may assume that $f_\nu>0$.
%   Let $k=\Vert\nu\Vert/\delta_\epsilon$ and let $f=f_\nu\wedge k\cdot f_\mu$.
%   Then

%   \ceq{\hfill 0 }{<}{f_\nu - f\medrel{<}\epsilon}

% \end{proof}
\clearpage



\begin{comment}
%%%%%%%%%%%%%%%%%%%%%%%%%%%%%%%%%%%%%
%%%%%%%%%%%%%%%%%%%%%%%%%%%%%%%%%%%%%
%%%%%%%%%%%%%%%%%%%%%%%%%%%%%%%%%%%%%
%%%%%%%%%%%%%%%%%%%%%%%%%%%%%%%%%%%%%
%%%%%%%%%%%%%%%%%%%%%%%%%%%%%%%%%%%%%
\section{A version of Radon-Nikodym}\label{samples}

\def\ceq#1#2#3{\parbox[t]{15ex}{$\displaystyle #1$}\medrel{#2}$\displaystyle  #3$}

\noindent\llap{\Large\color{red}\fontencoding{U}\fontfamily{futs}\selectfont\char 66\relax\ }%
Stupidaggini

In this section the canonical expansion of a finite model $M$ is a $3$-sorted structure \emph{$\<M,\NN, \NN^M\>$.}
%
The language of $\<M,\NN, \NN^M\>$ is denoted (again) by $L'$.
%
The elements of $\NN^M$ are interpreted as a multisets and are called \emph{samples.}
%
If $s\in\NN^M$ and $s(a)=n$, the element $a$ occurs $n$-times in the multiset. 
\begin{itemize}
  \item[2'.] $L'$ has a function $f_\phi:\NN^{(M^{x})}\times M^{z}\to\NN$ for each formula $\phi(x\,;z)\in L'$ and every finite tuples of variables $x\,;z$ of the home sort. 
  %
  The interpretation of $f_\phi(r,b)$ is
  
  \ceq{\hfill f_\phi(s,b)}
  {=}
  {\sum_{a\in\phi(M,b)}s(a)}

  We will use the notation \emph{$\ns|\phi(s,b)|$\/} for $f_\phi(s,b)$ which is suggestive of is interpretation of cardinality in the multiset setting. We write \emph{$\ns|s|$\/} when $\phi(x\,;z)$ is the formula $x=x$.
\end{itemize}

We write that $r\le s$ if $r(a)\le s(a)$ for every $a\in\U^x$.

Let $\D\subseteq\U^x$ be a definable set.
%
We denote by \emph{$\Indicator_\D$} the indicator function of $\D$.
%
This may be viewed as a multiset.

We say that $\D$ is $s$-large if $\ns|s\cdot\Indicator_\D|\sim\ns|s|$, where $\sim$ may be either $\simlin$ or $\simpoly$ according to the context.
%
We write $r\ll_\Delta s$ if every $r$-large $\{\wedge\}\pmDelta$-definable set is also $s$-large.

\begin{theorem}
  For every  $r,s\in\ns\NN^{(\U^x)}$ the folowing are equivalent
  \begin{itemize}
    \item[1.]  $r\ll_\Delta s$
    \item[2.]  $r'\le s$ for some $r'$ that is $\Delta$-equivalent to $r$.
  \end{itemize}
\end{theorem}


Let $s\in\ns\NN^{(\U^x)}$

%
We define

\ceq{\hfill{\Pr}_s(\D)}
{=}
{\inf\Big\{\frac{m}{n}\ :\ m,n\in\NN\sm\{0\} \text{ such that } n\cdot\ns|s\cdot\Indicator_\D|\le m\cdot\ns|s|\Big\}},

where the infimum is taken in $\RR$.
%
It is immediate that ${\Pr}_s(\mbox{-})$ is a finite probability measure on the definable subsets of $\U$.
%
(By the Caratheodory Theorem it can be extended to a probability measure but this is not required here.)

We say that $r,t\in\ns\NN^{(\U^{x})}$ are \emph{$\Delta$-equivalent samples\/} if  ${\Pr}_r(\D)={\Pr}_t(\D)$ for every $\{\wedge\}\pmDelta$-definable set $\D\subseteq\U^x$.
% 
We claim that the set of samples $\Delta$-equivalent $r$ is type definable. 
%
Let $x^{\rm s}$ be a variable with the sort of $\ns\NN^{(\U^{x})}$

\ceq{\hfill\emph{$\Delta^{\rm\!s}$}}{=}{\Big\{m{\cdot}\ns|\psi(x^{\rm s})|\le n{\cdot}\ns|x^{\rm s}|\ :\  m,n\in\NN\ \mathrm{ and }\ \psi(x)\in\{{\wedge}\}\pmDelta\Big\}.}

The samples that realize the type $p(x^{\rm s})=\Delta^{\rm\!s}\mbox{-}\tp(r)$ are those that are $\Delta$-equivalent to $r$.


% \begin{lemma}
%   Let $r,s\in\ns\NN^\U$ and suppose that  for every $b\in\U^z$. 
%   %
%   Then there is a finite definable partition of $\P_1,\dots,\P_n\subseteq\U^z$ and $n_i\in\NN$ such that  $\ns|(\phi(r\cdot\P_i\,;b)|=n_i\cdot \ns|(\phi(s\cdot\P_i\,;b)|$ for every $b\in\U^z$. 
% \end{lemma}
  
\end{comment}

\end{document}
